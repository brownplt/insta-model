\documentclass[english,cleveref,submission]{programming}

%% Thesis:
%%  1. Static Python is sound, fast, and practical
%%  2. SP has unique features and restrictions for speed
%%  3. Gradual soundness is a feature: start with weak types and eventually migrate to checked types

% > Here’s the thing: I think this is the sort of thing an academic might say “oh,
% > this is too much work, it’d never work” But in fact it looks like they seem to
% > be making it work -- Shriram 2021-09-30

%% TODO before submission
%% - look out for PROGRESSIVE TYPES~\cite{pqk-onward-2012}, where changing annotations
%%   helps the static checker find new errors. ChkDict is one example.
%% - idea: formalism would be a SUBSET of Nom, so our burden of proof is low
%% - what are the idiomatic base SP types ... int or Int ?

%% TODO after submission, before camera-ready
%% - 

\newcommand{\shorturl}[2]{\href{#1#2}{#2}}
\newcommand{\SP}{Static Python}
\newcommand{\code}[1]{\texttt{#1}}
\newcommand{\defeq}{=}
\newcommand{\mfeq}{=}
\newcommand{\langmid}{\mathrel{\mathbf{\Big\vert}}}
\newenvironment{langarray}{\(\def\arraystretch{1.5}\begin{array}{l@{\hspace{2mm}}c@{\hspace{2mm}}l}}{\end{array}\)}

\newcommand{\typefont}[1]{\mathsf{#1}}
\newcommand{\paramtype}[2]{#1[#2]}
\newcommand{\sptype}{\typefont{T}}
\newcommand{\sptclass}{\typefont{C}}
\newcommand{\sptint}{\typefont{Int}}
\newcommand{\sptfloat}{\typefont{Float}}
\newcommand{\sptdyn}{\typefont{Dynamic}}
\newcommand{\sptobject}{\typefont{Object}}
\newcommand{\sptnone}{\typefont{NoneType}}
\newcommand{\sptinstanceof}[1]{\paramtype{\typefont{Instance}}{#1}}
\newcommand{\sptfun}[2]{#1 \rightarrow #2}
\newcommand{\sptunion}[2]{#1 \cup #2}
\newcommand{\sptoptional}[1]{\paramtype{\typefont{Optional}}{#1}}
\newcommand{\sptrawpydict}{\typefont{PyDict}}
\newcommand{\sptpydict}[2]{\paramtype{\sptrawpydict}{#1, #2}}
\newcommand{\sptchkdict}[2]{\paramtype{\typefont{ChkDict}}{#1, #2}}

\newcommand{\spexpr}{e}
\newcommand{\spvalue}{v}

\newcommand{\sprred}{\rightarrow^*}

\newcommand{\mfapply}[2]{#1\,(#2)}
\newcommand{\mffont}[1]{\mathit{#1}}
\newcommand{\mftypeF}[1]{\mfapply{\mffont{F}}{#1}}
\newcommand{\mfopt}[1]{\mfapply{\mffont{opt}}{#1}}

\newcommand{\sperror}{\mathrm{Error}}

%\overfullrule=1mm

%\usepackage{draftwatermark}
%\SetWatermarkText{DRAFT}
%\SetWatermarkScale{1}

%% BEGIN tobias pape 2021-11-06
\makeatletter
\newcommand*\abstractpart[1]{\unskip\par\noindent{\firamedium\color{P@GrayFG}{#1}}\enspace}
\makeatother
%% END

%\usepackage{cleveref}
\usepackage{amsthm}
\newtheorem{theorem}{Theorem}
\newtheorem{definition}{Definition}
\usepackage[backend=biber]{biblatex}
\addbibresource{bg.bib}

\begin{document}

\title{Gradual Soundness: Lessons from Static Python}
%% ... types make things fast?
%% ... gradual migratory progressive soundness

%\titlerunning{short title}

\author[a]{Kuang-Chen Lu}
\authorinfo{(\email{LuKuangchen1024@gmail.com}) is a PhD student at Brown University.}
\affiliation[a]{Brown University, Providence, RI, USA}
\author[a]{Ben Greenman}
\authorinfo{(\email{benjamin.l.greenman@gmail.com}) is a PLT member and a postdoc at Brown University.}
\author[b]{Carl Meyer}
\authorinfo{(\email{carljm@fb.com}) FILL}
\affiliation[b]{Facebook, Inc.}
\author[b]{Dino Viehland}
\authorinfo{(\email{dinoviehland@fb.com}) FILL}
\author[a]{Shriram Krishnamurthi}
\authorinfo{(\email{shriram@brown.edu}) is the Vice President of Programming Languages (no, not really) at Brown University.}

%\authorrunning{K-C. Lu, B. Greenman, C. Meyer, D. Viehland, S. Krishnamurthi}

\keywords{gradual typing, migratory typing}

\begin{CCSXML}
\end{CCSXML}
% \ccsdesc[100]{FILL}
%TODO mandatory: Please choose ACM 2012 classifications from https://dl.acm.org/ccs/ccs_flat.cfm 

\paperdetails{
  %% perspective options are: art, sciencetheoretical, scienceempirical, engineering.
  perspective=scienceempirical,
  %% State one or more areas, separated by a comma. (see 2.2)
  %% Please see list of areas in http://programming-journal.org/cfp/
  %% The list is open-ended, so use other areas if yours is/are not listed.
  area={Database programming, General-purpose programming, Program
    verification, Programming education},
  %% License options include: cc-by (default), cc-by-nc
  % license=cc-by,
}

\maketitle

\begin{abstract}
%  Soundness, migratory typing.
%  Novel run-time check strategy, combines concrete and transient for ergonomics.

  \let\paragraph\abstractpart

  \paragraph{Context}
  % What is the broad context of the work? What is
  % the importance of the general research area?
  Gradually-typed languages allow typed and untyped code to interoperate,
  but typically come with some kind of drawback.
  In some languages, the types are unreliable;
  in others, communication across type boundaries can be extremely expensive;
  and still others allow only limited forms of interoperability.
  The research community is actively seeking a sound, fast, and expressive
  approach to gradual typing.

  %% TODO acknowlegde the success of new language / tracing JIT approaches

  \paragraph{Inquiry}
  % What problem or question does the paper
  % address? How has this problem or question been
  % addressed by others (if at all)?
  This paper describes \SP{}, a language developed by engineers at Instagram
  that has proven itself sound, fast, and reasonably expressive in production.
  \SP{}'s approach to gradual types is essentially a programmer-tunable combination of
  the \emph{concrete}\/ and \emph{transient}\/ approaches from the literature.
  Concrete types provide soundness in an efficient but difficult-to-use way.
  Transient types are sound in a shallow sense and easier to use; they help
  to bridge the gap between untyped code and typed concrete code.

  \paragraph{Approach}
  % What was done that unveiled new knowledge?
  In collaboration with the \SP{} team, we have evaluated the language
  as it exists today and developed a model to capture the essence of its
  approach to gradual types.
  We drew upon personal communications, bug reports, and the \SP{}
  regression test suite.

  \paragraph{Knowledge}
  % What new facts were uncovered? If the
  % research was not results oriented, what new
  % capabilities are enabled by the work?
  Our main finding is that the \emph{gradual soundness}\/ that
  arises from a mix of concrete and transient types is an effective
  way to lower the maintenance cost of the concrete approach.
  On a more technical level, this paper contributes two artifacts:
  a model of \SP{} that passes all relevant tests from the \SP{} codebase
  and a performance evaluation of \SP{}.

  \paragraph{Grounding}
  % What argument, feasibility proof, artifacts,
  % or results and evaluation support this work?
  Our model of \SP{} is implemented in PLT Redex and tested against
  the \SP{} regression suite.
  This paper includes a small core of the model to convey the main ideas
  of the \SP{} approach and its soundness.
  Our performance claims are supported by data from both
  a public suite a microbenchmarks and the proprietary Instagram codebase.

  \paragraph{Importance}
  % Why does this work matter?
  \SP{} is the first language that improves performance across the board through
  the addition of sound types to an untyped language.
  %% Prior work on Nom and Dart 2 showed that concrete types are promising.
  Other language designers may wish to replicate its approach,
  especially those who maintain unsound type checkers for untyped languages
  such as JavaScript.

\end{abstract}


\section{Typing for Performance}
\label{s:intro}

Gradual typing has attracted significant interest as a solution to
the impasse between static and dynamic typing.
The premise is simple: let programmers introduce types in part of a
codebase and while leaving the rest untyped.
Run-time checks can in principle enforce the assumptions that
typed code makes about untyped components, thereby ensuring that
the types are sound no matter how the untyped code behaves.

Unfortunately, the high run-time cost of sound types has split
the gradual typing community.
Industry teams have developed innovative type systems that accommodate
untyped designs, but are unsound [CITE].
These systems intentionally check nothing at run-time when untyped values enter
typed code.
Academic teams have primarily focused on the theory of sound
gradual types, formulating correctness properties and studying ever-more-descriptive types [CITE].
A few academics have studied the cost of run-time checks
in detail [CITE] and proposed implementation methods [CITE],
compiler technology [CITE],
and even weakened semantics [CITE], but these efforts have not yet
decisively closed the performance gap.
The most promising attempt is the \emph{concrete} semantics
for gradual types [CITE kafka nom etc], but its low performance
overhead stems from severe restrictions on untyped code.
%% In other words, concrete types impose low performance costs
%% but high migration costs.
%% ... you can have any color as long as it's black -h.ford
%% ... concrete expresses few programs, but runs them all nicely
Whether developers can accept these restrictions is unclear;
indeed, they may prefer a whole-sale migration over an awkward
gradual transition.\footnote{Successful migrations of untyped codebases
to a typed language are rare, but not unheard of.
Twitter ported its server-side code from Ruby to Scala
and Dropbox moved its core sync engine from Python mypy to Rust: see
\shorturl{http://www.}{artima.com/scalazine/articles/twitter\_on\_scala.html} and
\shorturl{https://}{dropbox.tech/infrastructure/rewriting-the-heart-of-our-sync-engine}.}

In short, academic researchers are working to close the performance gap
without overly restricting the promise of gradual typing.
Industry researchers are sidestepping the problem with unsound types---for the most part.

This paper reports on a remarkable exception to the rule among industry-made gradual type systems.
The \SP{} team at Instagram has developed a \emph{sound} type system for a subset of
Python along with a runtime system that uses soundness to drive optimizations.
The language design is a variation of concrete types that eases its restrictions.
Instead of asking programmers to migrate from untyped code to sound concrete types
in one leap, \SP{} offers an intermediate step via shallow types that enforce
only the top-level shape of values.
Programmers can begin with shallow claims and gradually dial up soundness.
On a suite of benchmarks and production modules, Static Python out-performs untyped
Python across the board.


\paragraph*{Contributions}

\begin{itemize}
  \item
    Evidence that gradual soundness delivers Python-level performance at a low migration cost
    and faster performance as programmers gradually incorporate sound types.
    \SP{} can be applied to a codebase with no refactoring and marginal performance overhead.
    After refactoring to sound types, \SP{} performs well on a microbenchmark
    suite and in production.
  \item
    A formal description of the \SP{} core language and an analysis of its soundness
    and optimizations.
    Soundness is gradual in the sense that some types are fully sound and others validate
    only top-level shapes.
  \item
    Discussions of the expressiveness limitations in \SP{} relative to
    idealized gradual type systems, and of the programming discipline that Instagram
    follows in their gradually-sound codebase.
\end{itemize}


\paragraph*{Significance}

We have written this paper with two audiences in mind.
First, we want to encourage system-builders to reproduce the
\SP{} language design.
In particular, the maintainers of optionally-typed languages
may find that adding transient and concrete types is a low-overhead
way to turn types into reliable claims.
Second, we want to lend focus to researchers.
Some of the restrictions that \SP{} adopts may be useful to
ground theoretical work.
Other restrictions might be lifted by future research.


\section{\SP{} and the Cinder JIT}
\label{s:tour}
%% purpose = SP looks like Pyre, adds soundness to certain types

% {Language Pipeline}
% \begin{enumerate}
%   \item Pyre as an optional, but recommended pre-check. (Not necessary for the model.)
%   \item SP type checker
%   \item Custom bytecode, Cinder JIT
% \end{enumerate}

\SP{} is a type checker and bytecode compiler.
It validates programs written in ordinary Python syntax
and generates type-specific instructions for the Cinder runtime.
The type system can express common Python idioms and is guaranteed
sound through run-time checks, which in turn enable optimizations.
The type system also includes a dynamic type as an escape hatch to
untyped behavior.
An unannotated variable behaves exactly as it would in Python.

Cinder is an extension of CPython 3.8 that adds a method-based
JIT compiler, virtual method tables, and several bytecode instructions
to directly express type checks and type-based optimizations.
These enhancements take full advantage of \SP{} type information;
they are built on top of CPython so that untyped code runs on
the platform.

Using \SP{} and Cinder is as easy as installing a new version
of CPython.
The main executable compiles programs to bytecode on the fly,
same as CPython, allowing standard developer tools to work.
By contrast, the only effective way to get type-directed optimizations
without changing CPython is to use C extension modules,
which introduce an extra compilation step and get in the way
of tooling~(\cref{s:related}).

\begin{figure}
  %% TODO replace with a real program
%  \begin{minipage}[t]{0.5\columnwidth}
  \begin{verbatim}
    from __static__ import PyDict

    def f(x: PyDict[str, int]):
      return x["A"]

    f({"A": 1}) # ==> 1
  \end{verbatim}
%  \end{minipage}\begin{minipage}[t]{0.5\columnwidth}
%  \begin{verbatim}
%    from __static__ import PyDict
%
%    def f(x: PyDict[str, int]):
%      check_args(x, PyDict)
%      return cast(x["A"], int)
%
%    invoke_function(f, {"A": 1}) # ==> 1
%  \end{verbatim}
%  %  ....
%  %  CHECK_ARGS
%  %  ....
%  %  INVOKE_FUNCTION
%  \end{minipage}
%  \begin{verbatim}
%    from __static__ import CheckedDict
%
%    def f(x: CheckedDict[str, int]):
%      return x["A"]
%
%    f(CheckedDict({"A": 1}))
%  \end{verbatim}
  \caption{A first \SP{} program}
  \label{fig:sp-example}
\end{figure}

\Cref{fig:sp-example} presents a first example program.
It defines a function \code{f} and calls it from typed code.
The syntax is normal Python with type annotations.
Only the import statement makes special use of \SP{};
it imports a type \code{PyDict} that describes Python dictionaries.

Although this program is typed, a different untyped module is free
to import the function \code{f} and invoke it with any sort of
argument.
For this reason, the compiled version of \cref{fig:sp-example} must
include a few run-time casts:
\begin{itemize}
  \item
    The function \code{f} must check that its arguments match the \code{PyDict} domain annotation.
    \SP{} inserts a check that accepts any Python dictionary.
  \item
    Because the domain check does not validate the elements of an incoming dictionary,
    the dict access (\code{x["A"]}) must check that its result matches the function codomain.
\end{itemize}
These checks ensure a \emph{shape-level} notion of type soundness~(\cref{s:model}).
In short: if an expression has type \code{T} and reduces to a value, then the value
matches the top-level constructor, or shape, of the type.
The shape for a Python dictionary is merely \code{dict}.
Other types have deeper shapes; for example, \SP{} includes a \code{ChkDict}
type that predicts the keys and values in a special kind of dictionary.

Thanks to these checks, \SP{} can leverage type soundness to generate efficient code.
Because the call to \code{f} in \cref{fig:sp-example} appears in typed code,
there is no need to check that it sends type-correct inputs.
Thus, the compiled call is optimized to use a Cinder bytecode instruction
that skips the domain check and immediately enters the function body.

The checks and optimizations in this example illustrate the general balancing act
of \SP{}.
Type soundness requires run-time checks, but also enables optimizations.
The gamble is that the optizations help more that the checks hurt.


\subsection{Type System Highlights}

The \SP{} type system is a unique synthesis of ideas from the gradual typing literature
and prior work on types for Python.
Nom~\cite{mt-oopsla-2017,mt-oopsla-2021} and PEP 484~[CITE] are two notable sources.
Relative to these, the \SP{} takes some interesting turns
because of the engineering context at Instagram.
The following contextual points are important to bear in mind as we present
notable aspects of the type system:
\begin{itemize}
  \item
    Performance is the bottom line.
    Unless \SP{} helps production code run faster, the team will be disbanded
    and reallocated.
    %% 2021-12-14: too extreme? and what about soundness?
  \item
    %% Q. why is the python so nominal? style guide? java background?
    The codebase makes heavy use of first-order classes and objects.
    First-class classes, objects, and even functions are rare.
  \item
    Instagram is / had been using Cython to improve the performance of select modules.
    Cython compiles Python source code to C extension modules for
    a faster Python [CITE].
    Instagram had converted several performance-critical modules to Cython prior
    to starting work on \SP{}.
    \SP{} replaced them.
    (How does this help understanding type system highlights?)
  \item
    The Instagram codebase is already typed.
    Developers use Pyre [CITE], an optional type checker for Python,
    to write type annotations and to drive IDE tools (e.g., autocompletion).
\end{itemize}



\subsubsection{Type Dynamic}
\label{s:type-dynamic}

%% 2021-12-14:
%% - anything to say about Any being a "meta" type in Python typing module?
%% - 3 classes of code ... how does dynamic differ from untyped?

Like most gradual languages, \SP{} includes a dynamic type
that allows untyped expressions within a statically-typed context.
Whenever an expression or variable lacks a type annotation,
\SP{} uses the dynamic type as a default and skips most compile-time checks.
Every untyped Python program is thus a well-formed \SP{} program.
Furthermore, programmers can omit the annotations for specific
function arguments without losing the benefits of other type checks.
Freedom to omit annotations helps programmers convert modules to \SP{}
with little initial refactoring.

In addition to serving as a default, the dynamic type also replaces any undefined types.
For example, suppose that the type \code{X} is undefined in \SP{} and that a
program contains the annotation \code{ChkDict[int, X]}, which describes a
dictionary with integer keys and \code{X}-typed values.
\SP{} interprets this annotation as \code{ChkDict[int, dynamic]} and allows any sort
of values inside the dictionary at runtime.
To an outsider, this behavior might seem odd: why not reject undefined types with an error?
The reason is to reduce friction with the Pyre annotations that are already
widely-used at Instagram.
Pyre is a mature static type system that can check more types than \SP{} knows how to enforce
efficiently, e.g., the type \code{Union[t1, ..., tn]} of arbitrary-width unions.
By replacing such types with dynamic, \SP{} can accept Pyre code that it does not yet fully understand.

The dynamic type is not, however, the flexible dynamic type provided
by \emph{true} gradual languages~\cite{svcb-snapl-2015}.
Replacing part of a type with dynamic can lead to both compile-time errors
and run-time errors.
In other words, \SP{} satisfies neither the static nor the dynamic gradual guarantee.\footnote{We
assume a standard type precision relation. Of course, one could argue that these programs
do not break the gradual guarantees if type precision does not allow our uses of the dynamic type.}
\Cref{fig:gg-failure} presents two examples, one for each kind of failure:
\begin{itemize}
  \item
    \cref{f:gg-failure-stat} presents a fully-typed class and a partially-typed subclass.
    The subclass definition raises a compile-time error because it attempts to override
    the typed \code{m1} method with another method that returns the dynamic type.
  \item
    \cref{f:gg-failure-dyn} presents a variant of \cref{fig:sp-example} that
    uses the type \code{ChkDict} for concrete-typed dictionaries (\cref{s:checked-type}).
    It sends a dictionary with integer values to a function that expects
    dictionaries with dynamic values.
    At runtime, the function rejects this argument because its type is not an
    exact match.
\end{itemize}
In sum, these behaviors show that performance is paramount in \SP{}.
Static errors in dynamic-typed code are designed to protect the integrity of
type-based optimizations.
%% other examples?
The dynamic errors are a consequence of strict run-time constraints, which provide stronger guarantees
and are simpler to validate than lax constraints.
%% hmph, unclear. Spell it out!

\begin{figure}
  \begin{subfigure}[t]{0.5\columnwidth}
  \begin{verbatim}
    class A:
      def m1(self)->int:
        return 0

    class B(A):
      def m1(self):
        return 0
    # Error: type dynamic does not match int
  \end{verbatim}
    \caption{Removing a type in \code{B} breaks the static gradual guarantee}
    \label{f:gg-failure-stat}
  \end{subfigure}
  ~
  \begin{subfigure}[t]{0.5\columnwidth}
  \begin{verbatim}
    from __static__ import ChkDict

    def f(x: ChkDict[str, dyn]):
      return x["A"]

    d = ChkDict[str, int]({"A": 1})
    f(d)
    # Error: f expected ChkDict[str, dyn]
  \end{verbatim}
    \caption{Removing part of the type for \code{x} breaks the dynamic gradual guarantee}
    \label{f:gg-failure-dyn}
  \end{subfigure}
  \caption{\SP{} values performance over the gradual guarantees.}
  \label{fig:gg-failure}
\end{figure}


\subsubsection{Concrete Types and Shallow Types}
\label{s:checked-type}

%% more to say about the style of migration in SP?
[FILL]

Enabling sound types on a function has a ripple effect on callers.
All untyped callers are now forced to supply type-correct inputs; may call for
refactoring.

Enabling checked types has a huge ripple effect.
The worst case is changing an annotation to a expect a checked data structure:
\begin{enumerate}
  \item all callers must create a checked value by invoking the right constructor, and
  \item all clients of those callers can no longer expect an unchecked type.
\end{enumerate}
Programmers have to trace the annotation back to value-creation points, and then
ensure that all uses of those values are compatible.


\subsubsection{Mixed-Typed Class Hierarchies}

%% TODO very important, one of the main things IG does use!!!

[FILL]

Typed classes can inherit from untyped classes.
Untyped classes can inherit from typed classes.

Interesting because StrongScript disallowed.
(But Nom allowed right?! Maybe allowed in a funny way, possibly to support their Dyn.)



\subsubsection{Types for Python}

[FILL]

Which types accommodate Python code? Which from PEP typing etc.

Unions are static only,
except for unions to false because those are common and can be implemented efficiently.

    Optional(T) is the only union. Others turn into Dynamic.


\subsubsection{Other}

\begin{itemize}
  \item \textbf{Recursive Types}.
    Sadly this is normal for Python type checkers.
\end{itemize}

\subsubsection{First-Order Functions and Classes}

Incidentally, \SP{} is a first-order type system.

... no user-facing type for functions
 imports / exports follow transient

... ditto for classes (see below)

%  \item \textbf{First-order Typed Classes}.
%    In order to benefit from types, class definitions must be defined at the top level of a module.
%    Nested classes are supported by the language, but have type \code{Dynamic} according
%    to the type checker.
%
%  \item \textbf{Callable}.
%    The type system cannot yet express first-class functions.
%    The team plans to lift this restriction~(\cref{s:future}),
%    but there is little internal pressure to do so.


\subsubsection{Restrictions}

\begin{itemize}
  \item \textbf{No Overloading}.
    Methods cannot be overloaded. Each combination of a class
    and a name can refer to at most one method.
    This ensures that method resolution depends only on the
    type of the receiver object.
    % resolution does NOT depend on result type (T doesn't matter in `x:T = o.m()`)
    % Q. does resolution depend on args?

  %\item \textbf{Single Inheritance}.
  %  Multiple inheritance is not supported.
  %  But, it might someday be supported.

  \item \textbf{Compatible method overrides}.
    Methods may be overridden, either by \SP{} classes or
    untyped Python classes.
    Overrides that appear in \emph{the same} \SP{} file as the class definition, however,
    must supply a method whose type is a static subtype of the previous method.
    In particular, a parent method that returns type \code{int} cannot be
    overridden by a child method that returns type \code{Dynamic}.
    %% TODO what does this enable ... optimizations in the same file?

  % NOTE __ (dunder) methods may be overridden e.g. __getattribute__,
  % but the normal field access syntax skips any overrides (o.f)

  \item \textbf{Compatible field overrides}.
    Similarly, object fields and class variables may be overridden
    only with subtype-compatible values.
    In fact, \SP{} compiles fields using Python slots
    declarations (\code{\_\_slots\_\_}).
    This declaration prevent new fields from arising at runtime,
    but saves space and speeds up field accesses.


\end{itemize}


Type restrictions:

\begin{itemize}
  \item \textbf{Parameterized Classes}.
    At present, \SP{} treats all class parameters as type \code{Dynamic}.
    The team plans to lift this restriction (\cref{s:future}).

%  \item \textbf{Dynamic Checked Types}.
%    A \code{CheckedDict} cannot be instantiated with \code{Dynamic}
%    keys or values.
%    [FILL] no desire, and makes runtime checks much simpler.
%    %% NB can't do ChkDict[ChkDict[_,_],_] either but that's because ChkDict is unhashable for now

\end{itemize}




\subsection{Runtime System Highlights}

\begin{itemize}
  \item
    \textbf{VTables}
    Cinder adds virtual method tables to typed classes.
    ``The v-tables are only used for the \code{INVOKE\_METHOD} opcode, which
    the static compiler will only emit against a statically known static type.'' -Carl

%    No, there’s no overload resolution by arguments in Python. This difference
%    in behavior isn’t so much desired as just a consequence of making
%    `INVOKE_METHOD` optimizable. With a normal dynamic `CALL_FUNCTION`, first
%    the callable is placed on the stack, then the arguments, then there is a
%    `CALL_FUNCTION` (with number of args in oparg) to perform the call. This
%    means that first the callable is resolved, then the arguments. But with
%    `INVOKE_METHOD` we want to resolve the callable as part of the invoke
%    itself, since this gives us opportunity to inline-cache the target of the
%    call instead of always having to call something dynamic and unknown that’s
%    on the stack. So that necessarily implies that first the arguments are
%    resolved and placed on the stack, then the callable is resolved as part of
%    the `INVOKE_METHOD`. -Carl
%
%    2021-11-15: Dino says the JIT will check/upgrade INVOKE_METHOD to INVOKE_FUNCTION
%    2021-07-23: final classes get INVOKE_FUNCTION, not INVOKE_METHOD

  \item
    \textbf{Bytecode Optimization}
    (could use a full section)

  %\item
  %  \textbf{Shadow Frames}
  %  (CinderDoc/static_python.rst)
  %  magic import that signals to the Static
  %  Python compiler to enable “shadow frame” mode in the Cinder JIT. This
  %  improves performance of function calls by avoiding the creation of full
  %  Python frame objects until they are definitely needed (e.g. if an
  %  exception is raised.) In the future this should become default.

  \item
    \textbf{Checked Dict}
    Implementation too, to support the types.

\end{itemize}


\subsection{Additional Features}

% things we did not model ... is that worth making a distinction here?

strict modules

decorator to force inlining, for 1-line functions


\subsubsection{Primitive Types}

For performance-critical applications, \SP{} includes a \emph{distinct} set of primitive types
that describe booleans and sized numbers, e.g., \code{int64}, \code{uint64}, \code{double}.
Cinder implements these types with unboxed C values, which are much simpler and cheaper
than their Python counterparts.
\SP{} also includes two special datatypes, \code{Array} and \code{Vector}, that store primitives efficiently.
These are not themselves primitive.

Primitives are carefully limited by the type system.
Neither a module-level nor a closure-level variable may have a primitive
type---because untyped code can mutate such variables.
Primitive types are incompatible with the dynamic type,
and there are no implicit upcasts from primitives
to a matching Python type.
Even a conjunction disallows mixing; for example, the \code{and} operator
requires either two Python booleans or two primitive booleans.
The only way for Python values and primitive values to interact
is through dedicated conversion functions.
Cinder handles conversions at the boundaries between \SP{} code
and untyped code to avoid a cascade of modify-then-run refactorings, but nowhere else.
Within typed code, programmers must satisfy the type checker with
appropriate conversions.

% For now, programmers have to write and manage primitive types.
% In the future, a preprocessor might convert Python arithmetic to primitive arithmetic.

% https://github.com/facebookincubator/cinder/issues/52



\section{Model}
\label{s:model}
%% purpose = types, checks, and optimizations can be formalized and analyzed


\subsection{Goals and Limitations}

Our model of \SP{} includes the key aspects of its types,
statements, and values.
The model has two main goals:
illustrate the boundaries where typed and untyped code can mix,
and show how to soundly protect these boundaries.
Compared to other gradual languages, these boundaries are quite restrictive---which
partly explains \SP{} performance~(\cref{s:eval}).
Indeed, the model is nearly a subset of the model for Nom~\cite{mt-oopsla-2021},
hence we focus on explaining the gaps and omit formal proofs.

The model intentionally does not cover all of Python.
Some aspects of \SP{} are left out because they are straightforward to
handle soundly (FILL examples ... while loop?).
Omitting these lets us focus on the difficult corners.
Other Python features are left out because their \SP{} semantics is identical
to Python.
These include \code{eval}, first-class classes, multiple inheritance, and module-level cells.
\SP{} does not rely on types for these features to guide program transformations;
see \cref{s:impl} for further discussion.


\subsection{Surface Language}

\begin{figure}[t]
  %% ported types from redex model
  \begin{langarray}
    \sptype & \defeq &
      \sptdyn \langmid
      \sptnone \langmid
      \sptclass \langmid
      \sptinstanceof{\sptclass} \langmid
      \sptfun{\sptype}{\sptype} \langmid
      \sptunion{\sptclass}{\sptclass}
  \\
    \sptclass & \defeq &
      \sptint \langmid
      \sptfloat \langmid
      \sptobject \langmid
      \sptpydict{\sptclass}{\sptclass} \langmid
      \sptchkdict{\sptclass}{\sptclass} \langmid
      \mbox{(user defined class)}
  \end{langarray}

  \begin{center}\parbox{0.8\columnwidth}{
    Abbreviation: $\sptoptional{\sptclass} \defeq \sptunion{\sptclass}{\sptnone}$

    EXCEPT THAT $\sptoptional{\sptclass}$ may be a generic parameter just like any other class $\sptclass$

    Function types cannot expect or return other function types.
  }\end{center}

  \caption{Surface Types}
  \label{f:surface-types}
\end{figure}


\subsection{Evaluation Language, Optimizations}

\begin{figure}[t]
  \(
    \mftypeF{\sptype_0}
    \mfeq
    \left\{\begin{array}{ll}
      \sptrawpydict & \mbox{if $\sptype_0 = \sptpydict{\sptclass}{\sptclass}$}
    \\
      \sptype_0 & \mbox{otherwise}
    \end{array}\right.
  \)

  \caption{Surface Types to Evaluation Types}
  \label{f:surface-to-eval-types}
\end{figure}


\subsection{Properties}

\begin{theorem}[Type Soundness]
  If\ \(~\vdash \spexpr_0 : \sptype_0\)
  then one of the following holds:
  \begin{itemize}
    \item
      \(\spexpr_0 \sprred \spvalue_0
        \mbox{ and }
        \vdash \spvalue_0 : \mftypeF{\sptype_0}
      \)
    \item
      \(\spexpr_0\) diverges
    \item
      \(\spexpr_0 \sprred \sperror\)
  \end{itemize}
\end{theorem}

\begin{theorem}[Optimization Soundness]
  If\ \(~\vdash \spexpr_0 : \mftypeF{\sptype_0}\)
  then\ \(~\vdash \mfopt{\spexpr_0} : \mftypeF{\sptype_0}\)
\end{theorem}

%\begin{theorem}[Optimization Effectiveness]
%  FILL how do we know optimizations are useful rather than trivial?
%  %% - length of eval(e) vs. eval(opt(e)) might be longer (good) or shorter (bad)
%  %%   for general heuristic opts.
%  %% - is [opt(e) != e] true for interesting programs?
%\end{theorem}


\subsection{Bug Reports}

While formalizing Static Python, we submitted N bug reports to the language developers.
The developers acknowledged M of these bugs as issues to fix.

\begin{center}
  %% most filed by KC, a few by Ben
  \shorturl{https://}{github.com/facebookincubator/cinder/issues/created\_by/LuKC1024}
\end{center}


\section{Scaling to Python}
\label{s:impl}

%% TODO discuss optimizations here, not in sec 2

%% TODO SP side channels:
%% ?? eval,
%% the locals and globals dicts (the latter can be accessed through frame objects at the moment),
%% and mutable closure cells.

%% TODO built-in vs. user defined ... 
%% Carl: I don't think this is quite true the way it's phrased here? in general
%%   for SP classes we wrap non-static subclass override methods at runtime (in our
%%   vtable implementation) to enforce correct return type, so that we don't need
%%   exact types, we can trust that Liskov is not broken.
%%     The cases where we tend to be limited to exact types is in optimizing
%%   operations on builtin types, where they aren't really static types and
%%   methods called on them aren't going through SP vtables, so we don't have
%%   the vtable wrappers to enforce LSP, we just know the behavior of the exact
%%   builtin type

%% TODO wherever we discuss optimizations, also talk about de-slowdowns
%% Carl: not sure where this belongs exactly, but we go to a lot of work in the
%%   compiler to intelligently minimize runtime checks by only inserting them when
%%   the source value is of dynamic type (this includes also skipping CHECK_ARGS if
%%   the function was statically invoked by another static function and thus we
%%   know the compiler checked argument types)


[FILL]

Purpose: explain significant gaps between the model and the implementation.
E.g. what would the TypeScript team need to know before using our model
to add soundness?

\subsection{Dynamic Python Features}

\SP{} does not ascribe types to the following Python features.
These are not covered in our model because the implementation simply
assigns the dynamic type and lets the runtime treat them as untyped
Python code.
For each, we claim that the dynamic type is a reasonable choice;
accurate static types would be difficult to maintain.


\paragraph{First-Class Classes}

\SP{} does not attempt to type first-class classes:
partly because they see little use at Instagram (FILL how much?)
and partly because it is unclear how to incorporate them into the
nominal type system.
The straightforward but restrictive approach is to force code that uses a
first-class class to expect subtypes of a particular named static class.
Flatt et al.~\cite{fkf-popl-1998} propose a more-flexible approach, %(specifically for mixins)
but it requires a second layer of \emph{interface types} atop the nominal hierarchy.
MonNom~\cite{mt-oopsla-2021} uses interfaces in a similar way to accommodate
structural objects.
Adding an interface layer to \SP{} is an open question.

% in particular , Python's multiple inheritance would complicate the mixin story.
% Classic mixins rely on structural types~\cite{bc-oopsla-1990}.

%Notes on first-class classes being untyped:
%- mixin-based code tends to structrural types
%- nominality overly restrictive, get forced into <: hierarchy
%- but don't have NO IDEA for wat to do
%  + classes and mixins paper shows one idea,
%    ... need complicated mixin study??? (many sub objects)
%    built new type system to be mixin-aware, layered atop java
%    unsure if such extension works for SP
%    furthermore, class-and-mixins depends on Java single inheritance
%  + bracha cook mixin pattern
%  + tate muehlboeck structural object + interfaces


\paragraph{Multiple Inheritance}

%% https://docs.python.org/3/reference/datamodel.html
%% TODO check Dino email, confirm with C&D that layout conflict is unrelated to vtables

Python allows classes to inherit from a list of parents.
When resolving a method call to such a class, Python
dynamically traverses the parent list in a fixed order
seeking a first match.

Due to the dynamic method resolution order (MRO),
\SP{} does not track types for classes with multiple parents.
Method calls to these classes execute the same way as
in Python, with no type-directed optimizations to speed up dispatch.


\paragraph{Dynamic Execution}
%% TODO is eval a side channel, or safe?
%% TODO check that eval/exec args really are unoptimized / sound etc.
Results computed by calls to \code{eval} and \code{exec}
have the dynamic type.
Their inputs run without being rewritten
by the optimizer.

Studies of JavaScript and R have shown that many uses of dynamic execution
can be removed through simple adjustments~\cite{rhbv-ecoop-2011,gdkkv-oopsla-2021,mrmv-esop-2012}.
Assuming these findings carry over to Python, we recommend either similar
adjustments or the introduction of semantically-descriptive replacements
over attempting to type eval.


\paragraph{Module-Level Variables}

Any Python module can read and write to the module-level variables of another module.
When compiling a module, \SP{} therefore assumes that its module-level variables
may be modified by untyped code and assigns the dynamic type.

Cinder offers \emph{strict modules} as an alternative to the Python
semantics.\footnote{https://instagram-engineering.com/python-at-scale-strict-modules-c0bb9245c834}
If a programmer chooses to declare a module as strict, then its module-level varibles
are immutable and thus typeable.

Another potential solution is for the Cinder runtime to check that writes to module-level variables
preserve their types.
There are two downsides to this idea:
it will add some performance overhead,
and it will force developers to rewrite untyped code in order to fix any errors that arise.
%% Because of the latter concern, type dynamic is a very reasonable default.


\section{Evaluation}
\label{s:eval}

\subsection{Microbenchmarks}

richards etc.

%% https://github.com/facebookincubator/cinder/tree/cinder/3.8/Tools/benchmarks
%%
%% deltablue
%% fannkuch
%% nbody
%% nqueens
%% pystone
%% richards
%% richards_static
%%
%% UHOH except for richards_static, none of these have types!!!
%% Need to ask if C&D have other typed benchmarks around
%% ... if not, ask them to port these and measure and tune if needed.


\subsection{Production Experience}

%% 2021-12-15: C&D may have an SP off/on switch soon
%% - accumulated CPU savings since July 1 = ~2% total IG production CPU
%%   ... how does that number work?
%% - in the beginning (?) H1 got 1% from first conversions + 0.7% from all Cython (higher bar)
%%   - no Cython regressions, SP + JIT is good enough
%%   - some Cython still exists, imported as library code from another codebase
%% - rough measure: 0 to 25% per module
%%   ... but depends on nature of module
%%   ... doesn't measure the cooperative effect of static-to-static cross-module calls

%% - 530 SP modules
%%   - 400 from a codegen tool
%%   - 130 by hand
%%   - currently +~3/4 per week by hand

%% - pushback one time about removing MI
%% - otherwise, people happy with low-cost improvements

%% - review of 30 winning diffs "provided the bulk of our wins this half"
%%   most common changes, not in careful order, are:
%%   1) adjustments to tests to pass correct types instead of mock objects, as as
%%      not to fail type checks in SP code. This is an inherent and desired
%%      incompatibility; next half we plan to work on improving the Python mock
%%      framework to automatically create mocks of the correct types.
%%      - bg: lookup the dino email about mocks
%%   2) adjustments to mock assertions in tests about calls to adjust for the
%%      fact that SP currently converts all call arguments to be positional (we plan
%%      to fix this incompatibility by falling back to the original call arguments
%%      when we detect the target function has been patched.)
%%      - bg: ??? bit fuzzy about positional and others
%%   3) adjustments around class attributes vs instance attributes (Python allows
%%      this to be a bit fuzzy, where a class attribute can be a fallback default
%%      for an instance attribute if not set, but SP currently requires an attribute
%%      to be clearly one or the other, and we have more efficient access for
%%      instance attributes via LOAD_FIELD that we don’t have for class attributes.)
%%      - related to github #37 ?
%%   4) convert @classmethod that don’t actually need the class to @staticmethod,
%%      allowing for emitting direct INVOKE_FUNCTION instead of INVOKE_METHOD
%%   5) extract special types that our compiler doesn’t yet support/understand
%%      (classes decorated with @dataclass or inheriting from `enum.Enum`) out of
%%      the module we are converting to static. We have work in progress on
%%      supporting some Enum types, and plan to add a @dataclass intrinsic soon.
%%   6) Removing the use of super-dynamic Python features like `__setattr__` that
%%      we don’t support. So far in the cases where we’ve done this it hasn’t
%%      required a major rewrite.
%%      - bg: more details pls ,,, sketch the rewrite
%%   7) Removing the use of keyword-only arguments (bare `*`) in function
%%      signatures, since we don’t yet support calling such functions. We plan to
%%      support this next half.
%%   8) Convert Python integers and bools used in very hot code paths into SP
%%      machine primitives.
%%      - bg: focus on block, convert, no problem?
%%   9) Occasionally converting a regular Python list or dict to a CheckedList or
%%      CheckedDict.
%%      - bg: how occasional? worried about power of the "gradual soundness" claim
%%
%% common theme: gotta change more tests than code
%%
%% significant change: replaced `contextlib.ContextDecorator` with `__static__.ContextDecorator`
%%  removes an extra call layer
%%  helps with (timing [ context managers / decorators ])
%%
%% all of this is the low-hanging fruit, haven't tackled uses of metaclasses even though looks promising
%%  because the pay / payoff isn't yet attractive




\section{Related Work}
\label{s:related}

Nom~\cite{mt-oopsla-2017} and MonNom~\cite{mt-oopsla-2021} are foundational.

Full Monty~\cite{pmmwplck-oopsla-2013} core calculus for Python language, foundational for us.

Thorn~\cite{wnlov-popl-2010} and StrongScript~\cite{rzv-ecoop-2015} are related.

PyPy is another runtime for Python
Pycket is built on PyPy and improves both Typed Racket~\cite{bbst-oopsla-2017}
and Reticulated Python~\cite{vsc-dls-2019}.

Type systems for Python ...
mypy pyre pytype;
Reticulated;

Optimizing type systems for Python ...
Reticulated;
mypyc;
Developer experience is one of the main advantages of \SP{} over mypyc,
which implements type-directed optimizations for CPython
by compiling source code to C extension modules.



\section{Future Work}
\label{s:future}

%% Immediate future: fix adoption friction points
%% 2021-12-15: WHAT ARE THESE?

%% engineering: JIT profiling, instead of hand-requested JIT list
%% model: metaclasses

Adapt SP ideas to a new setting, test performance takeaways.

Improve SP with generic types, structural types (lambda), \ldots.

Use confined GT to relax ChkDict ... let programmers decide whether
the type should reject or convert untyped data.
Maybe a ChkDict function could compile to an "overloaded" version with
fast and slow paths for ChkDict vs normal dict.

Build an automatic migration tool for Checked data.

Build a static analysis that hoists transient annotations to an early, shared point.
Take care to give quality error messages.

Optimize Python integer operations (in addition to the ctypes).
Hard because syntax like \code{a + b} may be the result of either \code{a.\_\_add\_\_(b)}
or \code{b.\_\_radd\_\_(a)} depending on runtime types and behavior.


\section{Conclusion / Lessons / Design Principles}
\label{s:conclusion}

\subsection{Nominal, Checked Types}

CheckedDict enables strong type checks and optimizations, but is painful to use.
It is painful even though ChkDict supports the full dict API
(any context that uses a Python dict can use a ChkDict without code changes).

The problem is that the ChkDict type rejects all Python values.
It accepts only ChkDict values, built through a special constructor
that installs a tag for type tests and guards writes.
Suppose that \code{f} is an untyped dict function.
As is, it can process ChkDict but gets no benefit from type checks nor from optimizations.
Adding a ChkDict domain type enables optimizations within \code{f} but raises a non-local
problem: all callers of \code{f} must be sure to create a ChkDict.
These ChkDicts must also be monomorphic to match whatever type \code{f} chose.
It can take many edits to get a program running again after adding a ChkDict type.

Lessons:
\begin{itemize}
  \item
    Nominal types are not compatible with structual values, such as Python values.
    Programmers are forced to edit old code to use such types ... unless the recent Nom work has a better idea~\cite{mt-oopsla-2021}
  \item
    Monomorphic nominal types are even worse.
    Editing old code to fit their rigid constraints may not be feasible.

\end{itemize}




Sound types in Static Python give programmers a way to improve performance.
The change is not forced, not intrusive;
programmers can keep using unsound types when working toward a deadline.

Conjecture that soundness will find bugs too, and lead to more reliable products.
Too soon to say.


\subsection{Communication}

How did we communicate effectively?
Starting point was GT survey [CITE].
Moved forward with example programs, see emails and github issues.
Our formalism had to cover their unit tests.
FILL

%\acks{
%  Thanks to
%  Guido van Rossum for stimulating tweets.
%  This work was partly supported by the US National Science Foundation.
%  This research was also developed with funding from the Defense Advanced Research Projects Agency (DARPA) and the Air Force Research Laboratory (AFRL).
%  The views, opinions and/or findings expressed are those of the author and should not be interpreted as representing the official views or policies of the Department of Defense or the U.S.~Government.
%  Greenman received support from NSF grant 2030859 to the CRA for the \href{https://cifellows2020.org}{CIFellows} project.
%}


{\sloppy
\printbibliography
}

%\appendix

\end{document}
