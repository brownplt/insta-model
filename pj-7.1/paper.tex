\documentclass[english,cleveref,submission]{programming}

%% Thesis:
%%  1. Static Python is sound, fast, and practical
%%  2. SP has unique features and restrictions for speed
%%  3. Gradual soundness is a feature: start with weak types and eventually migrate to checked types

% > Here’s the thing: I think this is the sort of thing an academic might say “oh,
% > this is too much work, it’d never work” But in fact it looks like they seem to
% > be making it work -- Shriram 2021-09-30

%% TODO after submission, before camera-ready
%% - reword 'fields' to attribute/descriptor/property?
%% - 
%% - 
%% - 

\newcommand{\shorturl}[2]{\href{#1#2}{#2}}
\newcommand{\SP}{Static Python}
\newcommand{\JSPP}{JS\code{++}}
\newcommand{\PEP}{PEP~484}
\newcommand{\code}[1]{\texttt{#1}}
\newcommand{\bcinst}[1]{\code{#1}}
\newcommand{\defeq}{=}
\newcommand{\mfeq}{=}
\newcommand{\langmid}{~\mid~} %% {\mathrel{\mathbf{\Big\vert}}}
\newenvironment{langarray}{\(\begin{array}{l@{\hspace{2mm}}c@{\hspace{2mm}}l}}{\end{array}\)}
\newcommand{\ipara}[1]{\emph{#1}\/.}
\newcommand{\nnum}[1]{$N=#1$}
\newcommand{\totalnum}[1]{$#1$ total}
\newcommand{\pctnum}[1]{$#1\%$}

%% TODO 2022-01-25 recount tests
\newcommand{\numbenchmark}{3}
\newcommand{\numSPtest}{265}
\newcommand{\numSPtotal}{802} %% SP commit 6d61575 from Jan 9
\newcommand{\numSPdiff}{537} %% (\numSPtotal - \numSPtest)
\newcommand{\CPUchange}{3.7\%}

\newcommand{\spapp}[2]{#1\,(#2)}
\newcommand{\spann}[2]{#1\!:\!#2}
\newcommand{\bmname}[1]{\textbf{#1}}
\newcommand{\typefont}[1]{\mathsf{#1}}
\newcommand{\codefont}[1]{\emph{#1}}
\newcommand{\paramtype}[2]{#1[#2]}
\newcommand{\sptype}{\typefont{S}}
\newcommand{\spteval}{\typefont{T}}
\newcommand{\sptclass}{\typefont{C}}
\newcommand{\sptX}{\typefont{X}} %% undefined type for an early example
\newcommand{\sptint}{\typefont{Int}}
\newcommand{\sptstr}{\typefont{Str}}
\newcommand{\sptbool}{\typefont{Bool}}
\newcommand{\sptfloat}{\typefont{Float}}
\newcommand{\sptflout}{\typefont{Flout}}
\newcommand{\sptdyn}{\typefont{Dyn}}
\newcommand{\sptobject}{\typefont{Object}}
\newcommand{\sptnone}{\typefont{None}}
\newcommand{\sptinstanceof}[1]{\paramtype{\typefont{Instance}}{#1}}
\newcommand{\sptoptional}[1]{\paramtype{\typefont{Optional}}{#1}}
\newcommand{\sptset}[1]{\paramtype{\typefont{Set}}{#1}}
\newcommand{\sptunion}[1]{\paramtype{\typefont{Union}}{#1}}
\newcommand{\sptrawpydict}{\typefont{Dict}}
\newcommand{\sptrawchkdict}{\typefont{CheckedDict}} %% not a real SP type, but useful for TeX
\newcommand{\sptpydict}[2]{\paramtype{\sptrawpydict}{#1, #2}}
\newcommand{\sptchkdict}[2]{\paramtype{\sptrawchkdict}{#1, #2}}
\newcommand{\sptenv}{\Gamma} %{\typefont{Env}}
\newcommand{\sptenvnil}{\cdot}
\newcommand{\sptenvempty}{\sptenvnil}
\newcommand{\sptvardef}[2]{\spann{#1}{#2}}
\newcommand{\sptfundef}[3]{\spapp{#1}{#2}\code{ -> }#3}
\newcommand{\sptclassdef}[4]{\mathrm{class}~\spapp{#1}{#2}:~#3;~#4}
\newcommand{\trule}[1]{\textsc{#1}}

\newcommand{\spx}{\code{x}}
\newcommand{\spf}{\code{f}}
\newcommand{\spc}{\sptclass}
\newcommand{\spprog}{\codefont{prog}}
\newcommand{\spstmt}{\codefont{stmt}}
\newcommand{\spexpr}{\codefont{expr}}
\newcommand{\spvalue}{\codefont{val}}
\newcommand{\vdashsub}[1]{\vdash_{#1}}
\newcommand{\wtprog}{\vdashsub{\mathbf{P}}}
\newcommand{\wtexpr}{\vdashsub{\mathbf{E}}}
\newcommand{\wtsub}{\vdash}
\newcommand{\spsubt}{\mathrel{<:}}
\newcommand{\spsubteq}{\mathrel{\leq:}}
\newcommand{\spcompat}{\sqsubseteq}
\newcommand{\spconsist}{\spcompat}
\newcommand{\spmatr}{\prec}
\newcommand{\spvardef}[3]{\sptvardef{#1}{#2} = #3}
\newcommand{\spfundef}[4]{\mathrm{def}~\spapp{#1}{#2}\code{ -> }#3: #4}
\newcommand{\spclassdef}[4]{\mathrm{class}~\spapp{#1}{#2}:~#3;~#4}
\newcommand{\spself}{\code{self}}
\newcommand{\spobject}{\code{object}}
\newcommand{\spnone}{\code{none}}
\newcommand{\spint}{\codefont{int}}
\newcommand{\spbool}{\codefont{bool}}
\newcommand{\spstr}{\codefont{str}}
\newcommand{\spfloat}{\codefont{float}}
\newcommand{\sppydict}[1]{\code{\{}#1\code{\}}}
\newcommand{\spchkdict}[3]{\spapp{\paramtype{\code{chkdict}}{#1, #2}}{#3}}
\newcommand{\spcnew}[2]{\spapp{#1}{#2}}
\newcommand{\spenvapp}[2]{\spapp{#1{}^*}{#2}}
\newcommand{\spdictref}[2]{#1[#2]}
\newcommand{\spdictset}[3]{\spdictref{#1}{#2} = #3}
\newcommand{\spobjref}[2]{#1.#2}
\newcommand{\spobjset}[3]{\spobjref{#1}{#2} = #3}
\newcommand{\spobjapp}[3]{\spobjref{#1}{\spapp{#2}{#3}}}

\newcommand{\sprred}{\rightarrow^*}

\newcommand{\mfapply}[2]{#1\,(#2)}
\newcommand{\mffont}[1]{\mathit{#1}}
\newcommand{\mftypeF}[1]{\mfapply{\mffont{R}}{#1}} % {\mfapply{\mffont{T}_{\downarrow}\!\mffont{S}}{#1}}
\newcommand{\mfcast}[2]{\mfapply{\mffont{Cast}}{#1, #2}}
\newcommand{\mfopt}[1]{\mfapply{\mffont{opt}}{#1}}

\newcommand{\sperror}{\mathrm{Error}}

\overfullrule=1mm

%\usepackage{draftwatermark}
%\SetWatermarkText{DRAFT}
%\SetWatermarkScale{1}

%% BEGIN tobias pape 2021-11-06
\makeatletter
\newcommand*\abstractpart[1]{\leavevmode\unskip\par\noindent{\firamedium\color{P@GrayFG}{#1}}\enspace}
\makeatother
%% END

\usepackage{tikz}
\usepackage{amsthm}
\usepackage{mathpartir}
\newtheorem{theorem}{Theorem}
\newtheorem{lemma}{Lemma}
\newtheorem{definition}{Definition}
\usepackage[backend=biber]{biblatex}
\addbibresource{bg.bib}

\begin{document}

\title{Gradual Soundness: Lessons from Static Python}
%% ... types make things fast?
%% ... gradual migratory progressive soundness

\author[a]{Kuang-Chen Lu}
\authorinfo{(\email{LuKuangchen1024@gmail.com}) } %% { is a PhD student at Brown University.}
\affiliation[a]{Brown University, Providence, RI, USA}
\author[a]{Ben Greenman}
\authorinfo{(\email{benjamin.l.greenman@gmail.com}) } %% {is a PLT member and a postdoc at Brown University.}
\author[b]{Carl Meyer}
\authorinfo{(\email{carljm@fb.com}) } %% FILL
\affiliation[b]{Meta, Menlo Park, CA, USA}
\author[b]{Dino Viehland}
\authorinfo{(\email{dinoviehland@fb.com}) } %% FILL
\author[b]{Aniket Panse}
\authorinfo{(\email{aniketpanse@fb.com}) } %% FILL
\author[a]{Shriram Krishnamurthi}
\authorinfo{(\email{shriram@gmail.com}) } %% {is the Vice President of Programming Languages (no, not really) at Brown University.}

\titlerunning{Draft document---please do not distribute or cite!}
\authorrunning{Draft document---please do not distribute or cite!}

\keywords{gradual typing, migratory typing, language design, concrete types, shallow types}

\begin{CCSXML}
\end{CCSXML}
% \ccsdesc[100]{FILL}
%TODO mandatory: Please choose ACM 2012 classifications from https://dl.acm.org/ccs/ccs_flat.cfm 

\paperdetails{
  %% perspective options are: art, sciencetheoretical, scienceempirical, engineering.
  perspective=scienceempirical,
  %% State one or more areas, separated by a comma. (see 2.2)
  %% Please see list of areas in http://programming-journal.org/cfp/
  %% The list is open-ended, so use other areas if yours is/are not listed.
  area={General-purpose programming},
  %% License options include: cc-by (default), cc-by-nc
  % license=cc-by,
}

\maketitle

\begin{abstract}
  \let\paragraph\abstractpart

  \vglue10pt

  \paragraph{Context}
  % What is the broad context of the work? What is
  % the importance of the general research area?
  Gradually-typed languages allow typed and untyped code to interoperate,
  but typically come with significant drawbacks.
  In some languages, the types are unreliable;
  in others, communication across type boundaries can be extremely expensive;
  and still others allow only limited forms of interoperability.
  The research community is actively seeking a sound, fast, and expressive
  approach to gradual typing.

  \vglue10pt

  \paragraph{Inquiry}
  % What problem or question does the paper
  % address? How has this problem or question been
  % addressed by others (if at all)?
  This paper describes \SP{}, a language developed by engineers at Instagram
  that has proven itself sound, fast, and reasonably expressive in production.
  \SP{}'s approach to gradual types is essentially a programmer-tunable combination of
  the \emph{concrete\/} and \emph{transient\/} approaches from the literature.
  Concrete types provide full soundness and low performance overhead, but impose
  nonlocal constraints.
  Transient types are sound in a shallow sense and easier to use; they help
  to bridge the gap between untyped code and typed concrete code.

  \vglue10pt

  \paragraph{Approach}
  % What was done that unveiled new knowledge?
  We evaluate the language in its current state
  and develop a model that captures the essence of its
  approach to gradual types.
  We draw upon personal communication, bug reports, and the \SP{}
  regression test suite to develop this model.

  \vglue10pt

  \paragraph{Knowledge}
  % What new facts were uncovered? If the
  % research was not results oriented, what new
  % capabilities are enabled by the work?
  Our main finding is that the \emph{gradual soundness\/} that
  arises from a mix of concrete and transient types is an effective
  way to lower the maintenance cost of the concrete approach.
  We also find that method-based JIT technology can eliminate the
  costs of the transient approach.
  On a more technical level, this paper describes two contributions: a model of
  \SP{} and a performance evaluation of \SP{}. The process of formalization
  found several errors in the implementation, including fatal errors.

  \vglue10pt

  \paragraph{Grounding}
  % What argument, feasibility proof, artifacts,
  % or results and evaluation support this work?
  Our model of \SP{} is implemented in PLT Redex and tested using
  property-based soundness tests and \numSPtest{} tests from the \SP{} regression suite.
  This paper includes a small core of the model to convey the main ideas
  of the \SP{} approach and its soundness.
  Our performance claims are based on production experience in the Instagram
  web server.
  Migrations to \SP{} in the server have caused a \CPUchange{} increase in
  requests handled per second at maximum CPU load.

  \vglue10pt

  \paragraph{Importance}
  % Why does this work matter?
  \SP{} is the first sound gradual language whose piece-meal application
  to a realistic codebase has consistently improved performance.
  %% Prior work on Nom and Dart 2 showed that concrete types are promising.
  Other language designers may wish to replicate its approach,
  especially those who currently maintain unsound gradual languages and are
  seeking a path to soundness.

\end{abstract}


\section{Introduction}
%% alt title:
%%  {A Staged Gradual Language}
%%  {Typing for Performance}
\label{s:intro}

Gradual typing has attracted significant interest as a solution to
the debate between static and dynamic typing.
The premise is simple: let programmers introduce types in part of a
codebase while leaving the rest untyped.
Run-time casts and checks enforce the assumptions that typed code makes about
untyped components, thereby ensuring that the types are sound no matter how
untyped code behaves.

Unfortunately, the high run-time cost of sound types has split
the gradual typing community.
Industry teams have developed innovative \emph{optional} type systems that accommodate
untyped designs, but are unsound~\cite{bat-ecoop-2014,rch-popl-2012,cvgrl-oopsla-2017,pep484}.
These systems intentionally check nothing at run-time when untyped values enter
typed code.
Academic teams have primarily focused on the theory of sound
gradual types, formulating correctness properties and studying ever-more-descriptive
types~(e.g.,~\cite{sgt-jfp-2016,clps-popl-2019,nla-popl-2019,mgt-oopsla-2021})
A few academics have studied the cost of run-time checks
in detail~\cite{tfgnvf-popl-2016,gtnffvf-jfp-2019}
and proposed implementation methods~\cite{kas-pldi-2019,fgsfs-oopsla-2018},
compiler technology~\cite{vsc-dls-2019,bbst-oopsla-2017},
and even weakened semantics~\cite{glfd-pj-2022,vss-popl-2017,gi-scp-2020},
but these efforts have not yet decisively closed the performance gap.
The most promising attempt is the \emph{concrete\/} semantics
for gradual types~\cite{clzv-ecoop-2018,mt-oopsla-2017,wzlov-popl-2010}, but it imposes nonlocal
constraints on untyped code.
In particular, concrete-typed client code is incompatible with
values created by untyped code.

In short, academic researchers are working to close the performance gap
without overly restricting the promise of gradual typing.
Industry researchers are sidestepping the problem with unsound types---for the most part.

This paper reports an exception to the rule among industry-made gradual type systems.
The \SP{} team at Instagram has developed a sound type system for a subset of
Python along with a runtime system that uses soundness to drive optimizations.
The language is staged to let programmers choose between easy migrations and
full-strength optimizations, a design that we call \emph{gradual soundness}.
To a first approximation, there are three main stages:
\begin{enumerate}
  \item
    A full language of \emph{shallow types\/} that describe idiomatic
    Python code at a coarse granularity.
    Adapting an untyped module to use shallow types requires at most a few local code
    changes.
  \item
    A second level of \emph{concrete types\/} describes generic data structures
    that check the types of their elements.
    If programmers modify their code to build concrete structures instead of Python
    ones (a potentially nonlocal change), then \SP{} can perform deeper optimizations.
  \item
    A third set of progressive types enable further optimizations.
    In particular, \emph{primitive types\/} allow for unboxed arithmetic,
    which has been critical in practice.
\end{enumerate}

Over the past year, Instagram has been applying gradual soundness to its
primary web server monolith.
Although only a handful of modules use \SP{} types (hundreds among thousands),
and only a few critical modules rely on concrete types and primitive types (dozens),
these migrations have resulted in a \CPUchange{} increase in production requests handled
per second for servers running at maximum CPU load.
These results are very positive:
Type-directed optimizations outweigh the cost of enforcing soundness.
Consequently, we believe the details of the \SP{} approach are of interest to
the gradual typing community at large.


\paragraph*{Contributions}
This paper makes two contributions:

\begin{itemize}
  \item
    \emph{Evidence}.
    We present evidence that \SP{} improves performance with few code changes.
    At Instagram, the web server has become significantly more efficient
    after the application of \SP{} to high-profile modules.
    Because the server code is closed-source, we additionally present data for
    \numbenchmark{} public microbenchmarks~(\cref{s:microbenchmarks}).
    Using only shallow types, \SP{} runs slightly faster than Python
    despite the costs of enforcing soundness~(avg. \pctnum{20}).
    With fine-tuned types, \SP{} consistently outperforms the Python
    baseline~(avg. \pctnum{70}).

  \item
    \emph{Mechanization}.
    To validate the soundness of \SP{}, we mechanized a core language in PLT Redex
    and ran both property-based soundness tests and over 200 tests adapted from the \SP{} regression suite.
    The modeling effort revealed 20 significant issues in \SP{},
    four of which were soundness bugs.

\end{itemize}

\paragraph*{Outline}

This paper begins with an informal description of \SP{} in two parts.
First, we present a user-oriented summary of the language~(\cref{s:tour}).
Second, we present the key ingredients of the runtime system, Cinder,
that supports \SP{}~(\cref{s:runtime}).
\Cref{s:model} uses a small formal model to introduce our Redex mechanization
and to convince readers that \SP{} is based on a sound core.
\Cref{s:impl} notes important aspects of \SP{} that are not covered by the model.
\Cref{s:eval} evaluates \SP{}; it reports our experience with the language in
production and on public microbenchmarks.
The paper concludes with related work~(\cref{s:related})
and a final summary~(\cref{s:conclusion}).


\paragraph*{Significance}

We have written this paper with two audiences in mind.
First, we want to encourage system-builders to reproduce the
\SP{} language design.
In particular, the maintainers of optionally-typed languages
may wish to focus on shallow types and JIT compilation as a
first step toward sound gradual types.
Second, we want to entice researchers.
The \SP{} type system has many noteworthy restrictions.
For example, functions are supported only by shallow types
and method overrides that use the dynamic type are more constrained
than overrides in untyped code.
Some of these restrictions might be lifted by future research.
Others might be useful to adopt in new language designs.


\section{A Tour of \SP{}}
\label{s:tour}

% {Language Pipeline for IG}
% \begin{enumerate}
%   \item Pyre as an optional, but recommended pre-check. (Specific to IG. Not necessary for the model.)
%   \item SP type checker
%   \item Custom bytecode, Cinder JIT
% \end{enumerate}

\SP{} is part of a large codebase that includes a type system, a tailored
bytecode, and a method-based JIT compiler.
In essence, \SP{} is the type system for an entirely new language.
The interface that it offers to programmers, however, replicates the
standard Python experience.
\SP{} runs Python 3.8 programs with minimal changes to their
behavior~(\cref{s:not-python}),
and it compiles code on-the-fly to support existing
IDEs and developer tools.
The advanced features of the \SP{} type system are offered on an opt-in basis
and arranged so that programmers can begin adding types one module at a time.

\begin{figure}
  \begin{minipage}{0.15\columnwidth}\noindent\begin{lstlisting}

def f(x):
  return x["A"]

f({"A": 1})
  \end{lstlisting}
\end{minipage}\begin{minipage}{0.07\columnwidth}\(~~\rightarrow\!\!\!\!\)\end{minipage}\begin{minipage}{0.32\columnwidth}\noindent\begin{lstlisting}
from typing import Dict
def f(x: Dict[str, int]):
  return x["A"]

f({"A": 1})
  \end{lstlisting}
  \end{minipage}\begin{minipage}{0.07\columnwidth}\(~~\rightarrow\!\!\!\!\)\end{minipage}\begin{minipage}{0.38\columnwidth}\noindent\begin{lstlisting}
from __static__ import CheckedDict
def f(x: CheckedDict[str, int]):
  return x["A"]

f(CheckedDict[str, int]({"A": 1}))
  \end{lstlisting}
  \end{minipage}


  \caption{A first \SP{} program and two migrations}
  \label{fig:sp-example}
\end{figure}

\Cref{fig:sp-example} presents a first example program and two modified versions that
utilize \SP{} types.
The basic program defines a function \code{f} and calls it with a dictionary
value.
\SP{} can run this program as-is and even JIT-compile the function.
The other two versions use dictionary types:
\begin{itemize}
  \item
    The $\sptrawpydict$ type describes normal Python dictionaries in a \emph{shallow} sense.
    At compile-time, \SP{} uses this type to find basic logical errors.
    At run-time, \SP{} enforces the type by checking that all inputs to \code{f} are
    dictionary values.
    These checks enable optimizations within the function body.

  \item
    The $\sptrawchkdict$ type
    describes a \emph{concrete} dictionary data structure provided by \SP{}.
    Unlike Python dictionaries, these checked dictionaries are guaranteed to
    contain well-typed keys and values even if they escape to untyped code.
    As the body of \code{f} illustrates, the syntax for using a checked dict is standard.
    Creating a checked dict, however, requires a type-parameterized constructor call.
\end{itemize}
These typed versions demonstrate the multi-level nature of \SP{}.
At one level, there are types that describe standard Python values that can be added
to a program with little-to-no code changes.
At the next level, \SP{} offers special-purpose types with stronger guarantees that impose
nonlocal maintenance costs, such as requiring edits to constructor calls.
% (i.e., programmers may need to modify constructor calls).


\subsection{Type System Context and Design Goals}
\label{s:ts-context}

The \SP{} type system is a unique synthesis of ideas from the gradual typing literature
and prior work on types for Python.
Its syntax for types is based on the definitions in \PEP{}~\cite{pep484}.
Its static semantics for types is inspired by optional type checkers; in
particular, Pyre~\cite{pyre} and mypy~\cite{mypy}.
And its strategy for run-time checks is adapted from Nom~\cite{mt-oopsla-2017,mt-oopsla-2021},
a research language with compelling performance results.
\SP{}'s novelty comes from the following engineering constraints:
\begin{itemize}
  \item
    Performance is the bottom line.
    At the end of the day, \SP{} needs to make code run faster in production.
  \item
    Critical code often relies on first-order functions and objects,
    at least in the Instagram web server.
    Precise types for higher-order functions, for first-class classes, and even for
    locally-defined functions and classes are thus a low priority; instead, the
    dynamic type serves as a coarse approximation.
  \item
    The codebase in which \SP{} is being applied makes heavy use of
    \PEP{} type annotations and is regularly analyzed by Pyre.
    \SP{} is therefore compatible with the \PEP{} syntax to reduce the
    adoption burden.
    %% depth-first approach? (the alternatives are not very clear ... depth vs breadth?)
  \item
    A module-level granularity is acceptable.
    Once \SP{} is enabled for a module, it compiles all code in that module
    including expressions that have the dynamic type~(\cref{s:type-dynamic}).
    If this behavior is a problem for certain expressions, it is easy enough to move
    them to an untyped Python module.
\end{itemize}
%
In general, the types that \SP{} implements all enable significant optimizations
and can be validated quickly with casts.
Types that do not meet these criteria are deferred to the mature optional
type checkers that already exist for Python.
In fact, \SP{} has no short-term plans to support all \PEP{} types.

The core supported types describe basic data (integers, booleans, strings),
data structures (lists, dicts, promises),
and nominal classes.
Union types are tracked statically; for example, if \code{x} is an integer and
\code{y} is a string, then the expression \code{x or y} has a union type.
Unions also narrow down via type tests as a kind of occurrence
typing~\cite{tf-icfp-2010,gsk-esop-2011}.
Union types are not, however, generally supported at run-time~(\cref{s:eval-types}).
The only exception is binary unions with the Python \code{None}
type ($\sptunion{\sptype, \sptnone}$, or
$\sptoptional{\sptype}$); these unions are enforced at run-time.

Three other unsupported types bear special mention:
first-class class and object types~(\cref{s:inheritance});
first-class function (callable) types; and
recursive types.
Pyre and mypy have some support for the first two kinds, but no support for recursive types.
%% mypy rec.types = https://github.com/python/mypy/issues/731
%% nothing in pep 484 = https://www.python.org/dev/peps/pep-0484/


\subsection{Type Dynamic}
\label{s:type-dynamic}

\SP{} includes a dynamic type that allows untyped expressions within a
statically-typed context.
Whenever an expression or variable lacks a type annotation, \SP{} uses the
dynamic type as a default and skips most static checks.
The overall goal of this design is to let programmers add simple annotations to
part of a Python module while leaving the rest untyped.

In addition to unannotated positions, the dynamic type is also the default for
\PEP{} types that \SP{} does not yet understand.
For example, if existing code uses the type $\sptset{\sptint}$ for sets of integers,
then \SP{} replaces it with dynamic.
Thus, \SP{} focuses on types that it can soundly optimize without forcing
programmers to remove types that other systems (e.g., mypy) can check.

The behavior of the \SP{} dynamic type is subtle because dynamic code
faces more restrictions than untyped Python code.
Replacing part of a type with dynamic can lead to either a static error or
a run-time error.
\Cref{fig:gg-failure} presents two examples, one for each kind of failure:
%% \footnote{Technically, \SP{} does not even
%% have syntax for the dynamic type. The idiomatic way to ask for dynamic is to
%% omit a type annotation.}
%% \footnote{We
%% assume a standard type precision relation. Of course, one could argue that these programs
%% do not break the gradual guarantees for a type precision relation that is tailored to \SP{}.}
%%   %% tailored = describes exactly the migrations that SP allows
%% \footnote{In addition, C types~(\cref{s:c-types}) are incompatible with type dynamic.}
\begin{itemize}
  \item
    \Cref{f:gg-failure-stat} presents a fully-typed class and a partially-typed subclass.
    The subclass definition raises a compile-time error because it attempts to override
    the typed \code{m} method to return the dynamic type.

  \item
    \Cref{f:gg-failure-dyn} sends a checked dictionary with integer values to a
    function that expects dictionaries with dynamic values.
    At runtime, the function rejects this argument because the two types are
    not an exact match.
\end{itemize}
These extra restrictions make the \SP{} dynamic type different than the flexible
dynamic type provided by languages that satisfy the gradual guarantees~\cite{svcb-snapl-2015}.
But they also enable efficient run-time checks for generics and type-directed optimizations
for method calls.
Part of the reason \SP{} weighs these benefits more heavily than the gradual guarantees
is that programmers have to opt-in to a feature in order to risk the errors.
The static error above comes only after enabling the \SP{} compiler on both a parent
class and its subclass.
The dynamic error comes only after adopting a concrete type for checked dictionaries.
In general, \SP{} is less interested in guarantees about \emph{removing
an arbitrary annotation} and more interested in making sure that \emph{an
untyped module compiles with minimal code changes}.

\begin{figure}
  \begin{subfigure}[t]{0.48\columnwidth}
    \begin{lstlisting}
class A:
  def m(self) -> int:
    return 0

class B(A):
  def m(self):
    # Error: dynamic cannot override int
    return 0
  \end{lstlisting}
    \caption{SGG Violation: removing a type in the subclass \code{B} raises a static error}
    \label{f:gg-failure-stat}
  \end{subfigure}
  \hfill
  \begin{subfigure}[t]{0.48\columnwidth}
  \begin{lstlisting}
from __static__ import CheckedDict

def f(x: CheckedDict[str, dynamic]):
  return x["A"]

d = CheckedDict[str, int]({"A": 1})
f(d)
# Error: f expected CheckedDict[str, dynamic]
  \end{lstlisting}
    \caption{DGG Violation: removing part of the type for the parameter \code{x} raises a dynamic error}
    \label{f:gg-failure-dyn}
  \end{subfigure}
  \caption{\SP{} provides neither the static (SGG) nor the dynamic (DGG) gradual guarantees (assuming a standard type-precision relation)}
  \label{fig:gg-failure}
\end{figure}

%% TODO carl 2021-12-24 ... very important to introduce machine types with ChkDict ... emphasizes that the GG is not necessary, but only nice-to-have
%% > One area where we definitely violate the gradual guarantee and always will
%% > is primitive machine types. Unlike Python object types these do not flow to
%% > dynamic, and dynamic can never flow to them! Effectively they form a
%% > fully-typed subset of the type system that does not participate in gradual
%% > typing and can never occur in non-static code. So this violates GG because
%% > there are many examples where removing an annotation of a primitive machine
%% > type can cause static code to no longer type check.


\subsection{Concrete Types and Shallow Types}
\label{s:checked-type}

Every \SP{} type is paired with an efficient cast operation that runs in
constant time regardless of how large an incoming value is.
Unlike structural gradual type systems~\cite{gtnffvf-jfp-2019},
no cast traverses an incoming value and no cast creates a wrapper
to monitor future behaviors.
Some casts for generic types are, however, incomplete.
We call the types with incomplete casts \emph{shallow} types
and the types with full casts \emph{concrete} types, following
the \code{StrongScript} authors~\cite{rzv-ecoop-2015}.

First off, casts for non-generic nominal types are straightforward.
Every type has a name and every value has a tag that corresponds to the name.
In the worst case, a cast must examine the parents of a class to find a match,
but these traversals are bounded in length and typically short.

Generic types are more difficult to enforce because a tag by itself does not
describe their contents.
For example, suppose that a function expects inputs that match
the type \code{Dict[int,int]}; that is, dictionaries with integer keys and
integer values.
A constant-time check for Python dictionaries is not enough to protect the
function against dicts with ill-typed elements.
The \SP{} solution is to provide a second kind of dictionary that ensures
the validity of its elements.
These different values have different types:
\begin{enumerate}
  \item
    The \emph{shallow} type $\sptrawpydict$ gets enforced with a tag check
    for dict values that ignores elements.
    Consequently, a parameterized surface type such as $\sptpydict{K}{V}$
    does not promise anything about keys and values.
  \item
    The \emph{concrete} type $\sptchkdict{K}{V}$ gets enforced with a
    check for concrete dicts that have exactly the same key and value
    types.
    To implement such checks efficiently, \SP{} stores a run-time registry
    of instantiated concrete types.
    Every concrete dict carries a pointer into this registry to enable
    constant-time type validation, even for nested concrete types.
    %% Programmers must use a type-annotated contructor to make a concrete-typed checked dict.
\end{enumerate}
%
These options keep run-time costs low and let programmers choose between
concrete and shallow types.
This choice is one aspect of gradual soundness in \SP{}.

Often, the shallow types are more attractive because concrete types impose
nonlocal changes.
If one type changes to concrete, then several changes may need to happen:
all values that reach this type must be initialized as concrete,
all typed clients of these values must expect concrete values,
all values that reach those clients must be concrete,
and so on and so forth.

The upside of concrete types is that they enforce stronger type constraints.
These constraints can catch bugs and always lead to faster typed code.
Faster performance is not guaranteed in general, though, because concrete types must
check all writes from untyped contexts.
For comparison, \cref{t:shallow-vs-concrete} presents the worst-case costs of
shallow and concrete dict types.
Shallow types pay a constant-time cost whenever an untyped value enters typed code
and whenever a typed context reads from a shallow value.
Concrete types pay a potentially-high cost for constructors,
a constant-time cost for casts and for writes in untyped code,
and zero cost for reads.
If a value travels to untyped codes and receives many writes but few reads,
then the shallow types may run faster in the end.

Not all generic \SP{} types have shallow and concrete versions at present,
though they are coming soon.
In particular, user-defined classes do not have concrete support.
The shallow check for a user-defined class does, however, guarantee the types
of all non-generic fields and methods.

\begin{table}[t]
  \newcommand{\nocost}{\textrm{---}}
  \caption{Costs of using shallow and concrete dictionaries in typed (T) and untyped (U) contexts relative to the number (N) of dict elements. A dash (\nocost{}) means zero cost.}
  \label{t:shallow-vs-concrete}

  \centering
  \(\begin{array}{l@{}rr@{\qquad}rr@{~~}@{\qquad}r@{~~}r}
                   & \mbox{Constructor} & \mbox{Cast} & \mbox{T-Read} & \mbox{T-Write} & \mbox{U-Read} & \mbox{U-Write} \\\hline
    \sptrawpydict         &   \nocost{} &        O(1) &          O(1) &      \nocost{} &     \nocost{} &      \nocost{} \\
    \sptchkdict{K}{V}     &        O(N) &        O(1) &     \nocost{} &      \nocost{} &     \nocost{} &           O(1)
  \end{array}\)
\end{table}



\subsection{Gradual Class Hierarchies}
\label{s:inheritance}

One important feature of \SP{} is that class hierarchies can
mix typed and untyped code.
An untyped class can inherit from a typed one and vice-versa,
letting programmers add types to a single class independently of its ancestors
and children.

% st-ecoop-2007 has no inheritance, object is collection of members simple (Abadi--Cardelli)

Gradual class hierarchies are rarely studied in the literature, especially
for a language where truly untyped classes can enter the mix.
Thorn~\cite{wzlov-popl-2010} and SafeTS~\cite{rsfbv-popl-2015}, for example,
provide separate hierarchies for untyped and (gradually) typed classes.
The closest related work, for Nom~\cite{mt-oopsla-2017}, implements
a flexible design inspired by the gradual guarantees.
In particular, Nom subclasses can override any part of a signature with the
dynamic type.
\SP{} implements a simpler design that restricts types and
inheritance:

\begin{enumerate}
  \item
    To benefit from types, classes must be declared at the module top level
    and may have at most one parent.
    Nested class declarations, first-class classes, and classes with
    multiple parents default to un-optimized Python behavior~(\cref{s:impl}).

  \item
    Methods cannot be overloaded.
    This restriction comes from Python.

  \item
    Non-final methods and fields may be overridden in arbitrary ways by untyped code.
    In typed code, however, overrides cannot use less-precise types.
    For example, a method that returns a number cannot be overridden by a method
    that returns the dynamic type; \cref{f:gg-failure-stat} shows the static error
    that arises from such an override.

    % In fact, \SP{} compiles fields using Python slots
    % declarations (\code{\_\_slots\_\_}).
    %
    % NOTE __ (dunder) methods may be overridden e.g. __getattribute__,
    % but the normal field access syntax skips any overrides (o.f)

  \item
    Typed fields cannot be overridden by descriptors or properties (\code{@property}).

\end{enumerate}
%
With this context in mind, \SP{} keeps track of whether each class is typed or untyped.
Each typed class can furthermore assume that if a method has a precise (non-dynamic)
type, then all typed overrides are subtypes of this type.
\SP{} can therefore optimize dispatch from typed callers to typed methods
and use checked dispatch for other combinations.
\SP{} can also optimize reads from typed fields.
One extra step arises when an untyped method overrides a typed method.
In this case, \SP{} creates a wrapper around the overriding method to check
that it computes type-correct results.
% Because these wrappers perform a simple first-order check, they
% are supported by an efficiently implementation that cooperates with
% \SP{}
The wrappers are handled efficiently by a tailored implementation of vtables~(\cref{s:vtable}).


\subsection{Progressive Primitive Types}
\label{s:c-types}

%% 2021-12-29: TODO
%% Using the term “ctypes” for these might be confusing, since the Python stdlib
%% has a “ctypes” module that isn’t related to our primitive types. We do call the
%% base type for them CType in our compiler, but when discussing them we usually
%% call them “primitive types” or “machine types” — “C types” could also work but
%% I wouldn’t combine it into one word :)

For performance-critical applications, \SP{} includes a set of primitive types
that describe booleans and sized numbers.
Example types include \code{int64}, \code{uint64}, \code{double}.
There are also two special datatypes, \code{Array} and \code{Vector}, that store primitives.
The purpose of these types is to allow unboxed values and arithmetic at
runtime.
Consequently, their static semantics is a \emph{progressive}~\cite{pqk-onward-2012}
refinement over the semantics of basic Python types:
they permit fewer behaviors statically to enable the unboxing.
Operations such as \code{and} (boolean conjunction), for example, do not accept
a mix of primitive and non-primitive arguments.
These primitive types are another main aspect of gradual soundness.

Because performance is the top priority for primitive types, they
are intentionally incompatible with the dynamic type.
Casts between a primitive type and the dynamic type are rejected
statically---another violation of the gradual guarantee---to encourage the
spread of primitives in typed code.
At the boundaries to untyped Python code, however, \SP{} inserts casts.
These casts let programmers introduce primitive types in one module
without having to rewrite its untyped dependents.

% Neither a module-level nor a closure-level variable may have a primitive
% type---because untyped code can mutate such variables.

% For now, programmers have to write and manage primitive types.
% In the future, a preprocessor might convert Python arithmetic to primitive arithmetic.

% https://github.com/facebookincubator/cinder/issues/52


\subsection{Behavioral Changes to Python}
\label{s:not-python}

Relative to Python, \SP{} makes a few behavioral changes to enable
sound types and strong optimizations:
class attributes must be set in an \code{\_\_init\_\_} method to
ensure predictable layouts (all \PEP{} checkers agree on this);
typed attributes are resolved statically, and thus cannot be overridden with
descriptors or properties;
module-level variables cannot be mutated;
typed fields 
and
multiple inheritance of either typed classes or untyped children
of typed classes is forbidden.

Several other Python features are unsupported at present:
enums; the \code{@dataclass} decorator;
overrides of the \code{\_\_setattr\_\_} and \code{\_\_getattr\_\_} methods;
keyword arguments with default expressions (rather than values); and
calls to keyword functions that use \code{**kwargs} with either
a custom object or a dict with extra keys.
The \SP{} team plans to lift these restrictions in the future.


\section{Runtime System Highlights}
\label{s:runtime}
%% alt title: Runtime System Highlights
%% Non-Highlights:
%% - shadow frames = JIT feature, lite array to track Python call stack

%% 2022-01-12
%% Cinder implements primitives with unboxed C values, which are much simpler and cheaper
%% than their Python counterparts.
%%
%% Cinder handles conversions at the boundaries between \SP{} code
%% and untyped code to avoid a cascade of modify-then-run refactorings, but nowhere else.
%% Within typed code, programmers must satisfy the type checker with
%% appropriate conversions.

The \SP{} runtime system, formally known as Cinder, extends
CPython 3.8 in several ways to take full advantage of static types.
Cinder includes tailored bytecode instructions, virtual method tables,
concrete datatypes, a registry of typed modules, and a JIT compiler.
These main ingredients of Cinder may be of interest to other teams seeking to
add sound gradual typing to an existing language.
For example, Node developers may wish to fork V8 and experiment with
bytecode instructions that leverage sound static types.


\paragraph{Bytecode Instructions}

All CPython 3.8 instructions have the same behavior in Cinder.
Cinder adds instructions to help with one of three tasks:
expressing runtime checks,
initializing concrete-typed values,
or efficiently performing a standard action.
As an example of the third kind, the \code{FAST\_LEN} instruction quickly
computes the length of a built-in value, which helps to optimize for loops.
\SP{} uses type information to decide where this instruction is appropriate
and applies it as an optimization.

%% TODO 2022-01-16 clarify why fast_len is interesting at all:
%% Carl: Two parts to the answer here. In the early days of Static Python we were
%% focused on matching Cython in some very hot code paths, and a couple of those
%% did take length of containers. So we implemented some things like FAST_LEN that
%% may not be broadly applicable but we needed to be as fast as possible in those
%% few Cython conversions. But since then we also implemented a general
%% optimization for iteration of loops where the container is known to be an exact
%% list (or array or tuple IIRC): when we see e.g. `for item in mylist` and we
%% know statically that `mylist` is an exact list/tuple/array, instead of
%% generating the usual bytecode that goes through the dynamic iterator protocol,
%% we will generate more efficient bytecode that uses `FAST_LEN` to get the length
%% and then uses an incremented (primitive) index counter and our optimized
%% `SEQUENCE_GET` opcode (without bounds checking since in this case we can
%% guarantee the index counter can never be out of bounds). The visible semantics
%% remain the same but under the hood we use numeric indexing instead of the
%% iterator protocol. So this optimization is much more generally applicable, and
%% relies on FAST_LEN.
%%
%% Also I think FAST_LEN is a convenient example optimization for explication
%% purposes, because it is simple and easy to understand but it’s reasonably
%% representative of the general approach.



\paragraph{Virtual Method Tables}
\label{s:vtable}

%% vtables go in:
%% - typed classes
%% - builtin classes that SP treats as typed
%% - typed and untyped subtypes of classes with vtables

Cinder adds virtual method tables (vtables) to classes.
These tables help to speed up method dispatch relative to Python's
dynamic lookup.
Calls to static methods that appear in statically-typed
code use the vtable to find an address for the method.
If the method is part of a final class, then the call is further optimized
to a direct function-call jump.
%% Just-in-time compilation~(\cref{s:jit}) can also upgrade vtable lookups to
%% function calls.
(Both vtable lookup and direct jumps are supported by Cinder-specific bytecode
instructions.)

The implementation of vtables happens to be built on the Python \code{vectorcall}
protocol; it is not a from-scratch development.
Cinder vtables are further specialized to check untyped overrides of typed methods
using a wrapper implemented in C (rather than in Python) to reduce performance costs.

%% beware: with vtables, method resolution is different than Python b/c args. get
%%   resolved before the receiver (A.m(B) goes "A->B" in Py. and "B->A" in SP)
%%
%% > This difference in behavior isn’t so much desired as just a consequence of making
%% > `INVOKE_METHOD` optimizable. With a normal dynamic `CALL_FUNCTION`, first
%% > the callable is placed on the stack, then the arguments, then there is a
%% > `CALL_FUNCTION` (with number of args in oparg) to perform the call. This
%% > means that first the callable is resolved, then the arguments. But with
%% > `INVOKE_METHOD` we want to resolve the callable as part of the invoke
%% > itself, since this gives us opportunity to inline-cache the target of the
%% > call instead of always having to call something dynamic and unknown that’s
%% > on the stack. So that necessarily implies that first the arguments are
%% > resolved and placed on the stack, then the callable is resolved as part of
%% > the `INVOKE_METHOD`. -Carl


\paragraph{Concrete Data Structures}
\label{s:chkdict-impl}

The concrete versions of built-in data structures come with both a type and an
implementation.
The implementation provides the same interface as the built-in but uses a
type tag to reject certain operations.

For example, the type $\sptchkdict{K}{V}$ describes a concrete dictionary
with keys of type $K$ and values of type $V$.
The implementation has three main components:
\begin{itemize}
  \item
    The constructor requires two type parameters and a Python dictionary ($\sptrawpydict$).
    It checks that the elements of the dictionary match the key and value types.
  \item
    All untyped writes must be protected by casts.
    Every operation that mutates or extends a checked dictionary must
    validate any untyped arguments that it receives.
  \item
    The dictionary value stores a type tag to support casts from the dynamic
    type.
    A tag represents a three-part type $\sptchkdict{K}{V}$ and is implemented
    as a pointer into a global registry of instantiated types.
    When a concrete dictionary enters typed code from an untyped context,
    its tag must match the expected type exactly.
    For example, type $\sptchkdict{K}{V}$ matches the type
    $\sptchkdict{K'}{V'}$ only if $K$ is equal to $K'$
    and $V$ is equal to $V'$.
\end{itemize}
%
In general, other checked datatypes have the same three components:
a constructor, checked update functions, and a tag.
The tag is supported by a global registry of instantiated concrete types.


\paragraph{Classloader}
\label{s:classloader}

\SP{} keeps track of typed functions and typed classes with a specialized classloader.
At runtime, the classloader keeps a registry of typed objects.
The registry helps the bytecode reference objects and types via their module names.


\paragraph{Method-at-a-Time JIT}
\label{s:jit}

Cinder includes a JIT (just-in-time) compiler for its bytecode.
At the moment, programmers enable the JIT by supplying a list of method names
to the compiler.
Any method can be JIT-compiled whether or not it uses \SP{} types,
even though fewer types usually leads to fewer optimizations.


\section{Model}
\label{s:model}

To validate the design of \SP{}, we developed a model of the language in PLT
Redex~\cite{kcdeffmrtf-popl-2012}.
The model covers a substantial part of the Python language, including
assertions, loops, exception handlers, and delete statements.
It follows \SP{}'s approach to typing these features.
The model is available at:
\begin{center}
  \url{https://github.com/brownplt/insta-model}
\end{center}
Because there is no prior formalization of (all of) Python and of \SP{}, we cannot
\emph{verify} that the model matches them. Instead, we have applied
three methods to give confidence that our model matches reality.
First, we manually reviewed the model in depth---using our
own expertise---to look for non-conformance.  Second, 
we used Redex's property-based testing tools~\cite{kf-sfp-2009} to check
type soundness.
Specifically, we used Redex to generate well-formed terms (using its default
algorithm) and checked whether the well-typed terms that terminated within a
fixed time budget (1,200 expressions and 11,000 programs in a typical run)
produced only well-typed values and allowed errors.
% \footnote{Property-based testing helped us find and fix several issues in the
% model.  None of issues had implications for \SP{}'s soundness.}
Finally, we employed the well-established method of testing end-to-end
conformance with a test suite~\cite{gsk-ecoop-2010,gclpk-dls-2012,pmmwplck-oopsla-2013,bcfgmnss-popl-2014,fgpssmds-popl-2016}.
We translated \numSPtest{} tests from the \SP{} regression suite to the
syntax of the model and confirmed that the results
do match,
which suggests that the model conforms to actual \SP{}.

For most of the \numSPtest{} tests, the translation is automatic.
A few tests required hand-pruning to remove features that the model does not
handle (\totalnum{52}).
\SP{} has \numSPdiff{} other tests (\totalnum{\numSPtotal{}}) that we did not
use because they fall outside the scope of the model.
\Cref{s:impl} summarizes the out-of-scope features and \cref{a:banned-test} gives
a detailed categorization.

%% 2022-01-20 redex-check details
%% 1 no counterexamples in 10000 attempts
%%   found 1636 well-typed expressions.
%%   1278 of them reduce to a value.
%%   358 of them don't reduce to a value within the step limit.
%%
%% 2 no counterexamples in 100000 attempts
%%   found 11963 well-typed programs.
%%   11924 of them terminate.
%%   39 of them don't terminate within the step limit.


\paragraph{The Payoff: Issues Reported}

%% Carl 2022-04-07
%% Ok, I completed my review of open issues. Turns out #55 was fixed back in
%% December already, we just never closed it :) I found and closed a couple others
%% in the same situation. We are left with a total of 9 issues opened by you all,
%% 8 by KC and 1 by Ben:
%%
%%  https://github.com/facebookincubator/cinder/issues/created_by/LuKC1024
%%
%%  https://github.com/facebookincubator/cinder/issues/created_by/bennn
%%
%% #61 is trivial, I have a fix up and it should land today.
%%
%% Of the rest, I don’t think any are soundness issues. Most are cases where we
%% could just be stricter or issue better errors statically (#40, #49, #50, #51,
%% #52, #60 – #60 might already be fixed, #50 is about to land). #63 is a small
%% improvement to type refinement. #71 is a pretty nasty issue that prevents use
%% of CheckedDict with dynamic key or value type in certain positions that might
%% take us a while to fix. But I’m comfortable saying that none of these are
%% soundness issues. Let me know if your analysis differs.

The modelling process helped uncover several critical issues in \SP{}.
This is a very important payoff given that \SP{} is running in production at
Instagram.
Overall, we made 25 bug reports~(\cref{a:github-issues}) to the \SP{} team via GitHub.
Four of these were soundness issues, one of which we could exploit to raise
a segmentation fault.
All but one of these soundness bugs have been fixed.
Of the remaining issues, five were relatively minor; these dealt with confusing error messages
and incorrect tests.
The remaining 16 issues report bugs in language design and implementation.
The \SP{} team has acknowledged these as bugs by applying a specific GitHub label: \lstinline{sp-correctness}.

%% critical soundness issues:
%% https://github.com/facebookincubator/cinder/issues/55
%% https://github.com/facebookincubator/cinder/issues/53
%% https://github.com/facebookincubator/cinder/issues/39
%% https://github.com/facebookincubator/cinder/issues/36

To give one example issue, \SP{} incorrectly accepted the following program whereas our model
reported a type error:
%% \url{https://github.com/facebookincubator/cinder/issues/62}

\medskip
\begin{minipage}[t]{0.25\columnwidth}~
\end{minipage}\begin{minipage}[t]{0.4\columnwidth}
\begin{lstlisting}
  from typing import Optional

  def f(x: Optional[str]) -> str:
    while True:
      if x is None:
        break
      return x
\end{lstlisting}
%
%  print(f(None))
%  # None
\end{minipage}

\noindent{}The function \code{f} expects either a string or the \code{None} value and promises to return a string.
When called with \code{None}, however, the function breaks out of the \code{while} loop
and implicitly returns \code{None} contrary to the return type.
\SP{} had failed to account for the \code{break} and implicit return.
More concerningly, the associated test case \emph{expected} the program to type check.
Our model caught this specification error.

% In addition to these formal issue reports,
% we had several long discussions together (between Brown and Instagram)
% about finer points in the language design.

\paragraph{Section Outline}

%% TODO gaps between paper and redex
%% (if these are small enough, explain here ... maybe with minor notes further down
%%  if bigger, consider a "threats" or "gaps" section)
%% - typing is not algorithmic
%% - few python features (well that's obvious)
%% - few expressions (very subtle to introduce here/now)

The rest of this section illustrates our Redex model using a small
formalization.
%% TODO EXAMPLE simplification
\Cref{s:surface} presents a surface syntax to highlight the boundaries between
typed and untyped code.
\Cref{s:eval-types} explains how types get enforced at run-time.
\Cref{s:ts} argues that the overall approach toward boundaries is sound.


\subsection{Surface Syntax and Types}
\label{s:surface}

\Cref{f:surface-types} presents a core syntax for programs.
A program is a sequence of statements;
a statement defines a variable, function, or class.
These definitions may only appear on the top level
and they all require type annotations.
Functions must have one positional argument.
Classes must declare one parent (either $\sptobject$ or another class), one field, and one method.
Expressions describe values and simple computations.
The basic values are the none value, integers,
booleans (which are the integers \code{0} and \code{1}), strings, and the top object.
There are two data structures: dictionaries and checked dictionaries~(\cref{s:checked-type}).
Other expressions describe function calls, dictionary reads and writes, object field reads and writes,
and method calls.

In Python, the syntax for expressions overlaps with the syntax of types.
For example, the Python name \code{None} is both an expression and a type,
and the \SP{} names \code{chkdict} and \code{CheckedDict} are synonyms.
\Cref{f:surface-types} does not follow these standards; instead it keeps the two syntaxes distinct.
Expressions use lowercase names (\code{none}, \code{chkdict}) and types use capitalized
names (\code{None}, \code{CheckedDict}).

Types $\sptype$ include the dynamic type, types for the basic values, one type $\sptclass$ for every
user-defined class, and union types.
We choose to write the dynamic type as $\sptdyn$ for simplicity; technically,
\SP{} does not have a surface syntax for its dynamic
type and instead allows missing and unknown types~(\cref{s:type-dynamic}).
We assume that all unions are written in a flat and simplified form, e.g., that
$\sptunion{\sptint, \sptunion{\sptdyn, \sptclass_0}}$ would be normalized to $\sptdyn$.

\SP{} does not enforce all surface types against untyped code, only the evaluation
types presented in the next subsection~(\cref{s:eval-types}).
For this reason, we omit the surface typing judgment, which is a kind of linter
that merely scans typed code for logical errors.

Relative to \SP{} and our Redex mechanization, the formalization in \cref{f:surface-types} omits
many details of Python including class variables, imports, conditionals, and exception handlers.
These details are crucial in the mechanization, which tests whether \SP{}
soundly approximates Python.
They are less important here, where our focus is on type boundaries.
None of the omitted features give substantially new ways for typed code to interact with dynamic code.

\begin{figure}[t]
  \begin{langarray}
    \spprog & \defeq &
      \spstmt \langmid \spstmt,~\spprog
    \\[1.5ex]
    \spstmt & \defeq &
      \spvardef{\spx}{\sptype}{\spexpr} \langmid
      \spfundef{\spf}{\spann{\spx}{\sptype}}{\sptype}{\spexpr} \langmid
  \\ & &
      \spclassdef{\spc}{\spc}{\spvardef{\spx}{\sptype}{\spexpr}}{\spfundef{\spf}{\spself, \spann{\spx}{\sptype}}{\sptype}{\spexpr}}
    \\[1.5ex]
    \spexpr & \defeq &
      \spx \langmid \spnone \langmid \spint \langmid \spbool \langmid \spstr \langmid
      \spapp{\spf}{\spexpr} \langmid
  \\ & &
      \sppydict{\spx: \spexpr, \ldots} \langmid
      \spchkdict{\sptype}{\sptype}{\sppydict{\spx: \spexpr, \ldots}} \langmid
  \\ & &
      \spdictref{\spexpr}{\spexpr} \langmid
      \spdictset{\spexpr}{\spexpr}{\spexpr} \langmid
  \\ & &
      \spcnew{\sptclass}{\spexpr, \ldots} \langmid
      \spobjref{\spexpr}{\spx} \langmid
      \spobjset{\spexpr}{\spx}{\spexpr} \langmid
      \spobjapp{\spexpr}{\spf}{\spexpr}
    \\[1.5ex]
    \sptype & \defeq &
      \sptdyn \langmid
      \sptnone \langmid
      \sptint \langmid
      \sptbool \langmid
      \sptstr \langmid
      \sptclass \langmid
  \\ & &
      \sptpydict{\sptype}{\sptype} \langmid
      \sptchkdict{\sptype}{\sptype} \langmid
      \sptunion{\sptype, \ldots}
    \\[1.5ex]
    \sptenv & \defeq &
      \sptenvnil \langmid
      \sptvardef{\spx}{\sptype},~\sptenv \langmid
      \sptfundef{\spf}{\sptype}{\sptype},~\sptenv \langmid
      \sptclassdef{\spc}{\spc}{\sptvardef{\spx}{\sptype}}{\sptfundef{\spf}{\sptclass, \sptype}{\sptype}},~\sptenv
    \\[1.5ex]
    \spx, \spf & \defeq & \mbox{variable names}
  \end{langarray}

  \bigskip
  \mbox{Abbreviation: $\sptoptional{\sptype} \defeq \sptunion{\sptnone, \sptype}$}

  \medskip
  \mbox{Convention: the set of all class types $\sptclass$ contains the top type $\sptobject$}

  \caption{Surface syntax and types}
  \label{f:surface-types}
\end{figure}


\subsection{Evaluation Types, Casts, and Typing Judgment}
\label{s:eval-types}

\begin{figure}[t]
  \begin{langarray}
    \spteval & \defeq &
      \sptdyn \langmid
      \sptnone \langmid
      \sptint \langmid
      \sptbool \langmid
      \sptstr \langmid
      \sptclass \langmid
    \\ & &
      \sptrawpydict \langmid
      \sptchkdict{\spteval}{\spteval} \langmid
      \sptoptional{\spteval}
  \end{langarray}

  \bigskip
  \(
    \mftypeF{\sptype_0}
    \mfeq
    \left\{\begin{array}{ll}
      \sptrawpydict & \mbox{if $\sptype_0 = \sptpydict{\sptype}{\sptype}$}
    \\
      \sptoptional{\mftypeF{\sptype_1}} & \mbox{if $\sptype_0 = \sptoptional{\sptype_1}$}
    \\
      \sptdyn & \mbox{if $\sptype_0 = \sptunion{\sptype, \ldots}$ and $\sptype_0 \neq \sptoptional{\sptype}\!\!\!\!$}
    \\
      \sptchkdict{\mftypeF{\sptype_1}}{\mftypeF{\sptype_2}} & \mbox{if $\sptype_0 = \sptchkdict{\sptype_1}{\sptype_2}$}
    \\
      \sptype_0 & \mbox{otherwise}
    \end{array}\right.
  \)

  \caption{Evaluation types, Surface-to-Evaluation mapping}
  \label{f:surface-to-eval-types}
\end{figure}

Evaluation types $\spteval$ are a simplification of the surface types.
These are the types that \SP{} promises to soundly enforce.
\Cref{f:surface-to-eval-types} presents the syntax of evaluation types
and a retraction $\mftypeF{\cdot}$ from surface types to evaluation types.
As the retraction shows, the evaluation types make two main changes:
\begin{enumerate}
  \item
    The parameterized type for Python dictionaries
    $\sptpydict{\sptype_0}{\sptype_1}$ gets replaced with
    a raw type $\sptrawpydict$.
    The raw type behaves the same as $\sptpydict{\sptdyn}{\sptdyn}$ would.
  \item
    Unions get replaced with the dynamic type, except for the special case
    of unions with none ($\sptoptional{\sptype_0}$).
\end{enumerate}


\subsubsection{Casts}

\SP{} evaluation types can all be enforced fully and efficiently
with casts.
\Cref{t:cast} describes the casts in detail.
Most call for a tag check, i.e., a Python \code{isinstance} test.
The sole exception is optional types, which require a tag check and a test for the none value.
Even checked dictionaries rely on tag checks.
A checked dict type with keys $\spteval_0$ and values $\spteval_1$ accepts only
checked dict values that were initialized with exactly the same key and value type.
As noted in~\cref{s:type-dynamic}, this exact-match rule applies even when $\spteval_0$
or $\spteval_1$ is the dynamic type.

No cast requires traversing a data structure.
Similarly, no cast allocates a wrapper to check higher-order behaviors in a delayed fashion.
This latter property means that blame-tracking~\cite{ff-icfp-2002} is trivial because
casts either succeed or fail immediately.
In other words, \SP{} provides the same \emph{immediate accountability} as
Nom~\cite{mt-oopsla-2017}.

%% 2022-01-15: unclear if this is helpful or redundant, hiding for now
%The strict cast for checked dictionaries greatly simplifies type compatibility for \SP{}.
%Values that are partly-dynamic are simply incompatible with static types.
%For example, suppose that a function expects a list of numbers.
%In a typical gradual language, a list of dynamic values would be a valid input to the function.
%In \SP{}, a list of dynamic values is not a valid input.
%Consequently, \SP{} can use an ahead-of-time strategy to enforce soundness.
%Any position where the dynamic type meets a static type needs a cast.
%Other meetings between types can be resolved statically.
%\SP{} does not need to worry that a list of dynamic values might be masquerading
%under a fully-static type.

\begin{table}[t]
  \begin{threeparttable}
  \centering
  \caption{Description of run-time checks for the evaluation types}
  \label{t:cast}

    \begin{tabular}{l@{\quad}l}
    {Eval. Type $\spteval$} & {Run-time check} \\\hline
    $\sptoptional{\spteval_0}$ & {Accepts either the none value or values that match $\spteval_0$} \\
    $\sptrawpydict$ & {Accepts any Python dictionary} \\
    $\sptchkdict{\spteval_0}{\spteval_1}$ & {Accepts checked dictionaries parameterized by $\spteval_0$ and $\spteval_1$} \\
    $\sptdyn$ & {No check needed, accepts any value} \\
    $\sptclass_0$ & {Accepts instances of the class $\sptclass_0$\tnote{*}} \\
    $\sptnone$ & {Accepts the Python none value} \\
    $\sptint$ & {Accepts integers} \\
    $\sptbool$ & {Accepts booleans (the integers \code{0} and \code{1})} \\
    $\sptstr$ & {Accepts strings}
    \end{tabular}

    \begin{tablenotes}
    \item {*} {\footnotesize In \SP{}, the check for $\sptobject$ accepts all values except for primitives~(\cref{s:c-types})}
    \end{tablenotes}
\end{threeparttable}
\end{table}


\subsubsection{Expression Typing, Cast Insertion}

\begin{figure}[p]
  \fbox{$\sptenv \wtexpr \spexpr : \spteval$}~selected rules
  \begin{mathpar}
    \inferrule*[left=S-App]{
      \sptfundef{\spf_0}{\spteval_0}{\spteval_1} \in \sptenv
      \\\\
      \sptenv \wtexpr \spexpr_0 : \spteval_0
    }{
      \sptenv \wtexpr \spapp{\spf_0}{\spexpr_0} : \spteval_1
    }

    \inferrule*[left=D-App]{
      \sptvardef{\spf_0}{\sptdyn} \in \sptenv
      \\\\
      \sptenv \wtexpr \spexpr_0 : \sptobject
    }{
      \sptenv \wtexpr \spapp{\spf_0}{\spexpr_0} : \sptdyn
    }

    %% TODO show dict read rules?
%    \inferrule*[left=D-Ref]{
%      \sptenv \wtexpr \spexpr_0 : \sptrawpydict
%      \\\\
%      \sptenv \wtexpr \spexpr_1 : \sptobject
%    }{
%      \sptenv \wtexpr \spdictref{\spexpr_0}{\spexpr_1} : \sptdyn
%    }
%
%    \inferrule*[left=CD-Ref]{
%      \sptenv \wtexpr \spexpr_0 : \sptchkdict{\spteval_0}{\spteval_1}
%      \\\\
%      \sptenv \wtexpr \spexpr_1 : \spteval_0
%    }{
%      \sptenv \wtexpr \spdictref{\spexpr_0}{\spexpr_1} : \spteval_1
%    }

    \inferrule*[lab=D-Set]{
      \sptenv \wtexpr \spexpr_0 : \sptrawpydict
      \\\\
      \sptenv \wtexpr \spexpr_1 : \sptobject
      \\\\
      \sptenv \wtexpr \spexpr_2 : \sptobject
    }{
      \sptenv \wtexpr \spdictset{\spexpr_0}{\spexpr_1}{\spexpr_2} : \sptnone
    }

    \inferrule*[lab=CD-Set]{
      \sptenv \wtexpr \spexpr_0 : \sptchkdict{\spteval_0}{\spteval_1}
      \\\\
      \sptenv \wtexpr \spexpr_1 : \spteval_0
      \\\\
      \sptenv \wtexpr \spexpr_2 : \spteval_1
    }{
      \sptenv \wtexpr \spdictset{\spexpr_0}{\spexpr_1}{\spexpr_2} : \sptnone
    }

    %\inferrule*[left=F-Ref]{
    %  \spann{\spx_0}{\spteval_0} \in \spenvapp{\sptenv}{\spc_0}
    %}{
    %  \sptenv \wtexpr \spobjref{\spc_0}{\spx_0} : \spteval_0
    %}

    \inferrule*[left=F-Set]{
      \sptenv \wtexpr \spexpr_0 : \spc_0
      \\\\
      \spann{\spx_0}{\spteval_0} \in \spenvapp{\sptenv}{\spc_0}
      \\\\
      \sptenv \wtexpr \spexpr_1 : \spteval_0
    }{
      \sptenv \wtexpr \spobjset{\spexpr_0}{\spx_0}{\spexpr_1} : \sptnone
    }

    \inferrule*[left=M-App]{
      \sptenv \wtexpr \spexpr_0 : \spc_0
      \\\\
      \sptfundef{\spf_0}{\spc_0, \spteval_0}{\spteval_1} \in \spenvapp{\sptenv}{\spc_0}
      \\\\
      \sptenv \wtexpr \spexpr_1 : \spteval_0
    }{
      \sptenv \wtexpr \spobjapp{\spexpr_0}{\spf_0}{\spexpr_1} : \spteval_1
    }

    \inferrule*[left=C-Sub]{
      \sptenv \wtexpr \spexpr_0 : \spteval_1
      \\\\
      \sptenv \wtsub \spteval_1 \spcompat \spteval_0
    }{
      \sptenv \wtexpr \spexpr_0 : \spteval_0
    }

    \inferrule*[left=Matr]{
      \sptenv \wtexpr \spexpr_0 : \spteval_1
      \\\\
      \spteval_1 \spmatr \spteval_0
      %% side condition: result type is not a primitive
    }{
      \sptenv \wtexpr \spexpr_0 : \spteval_0
    }

  \mbox{Where $\spenvapp{\sptenv}{\spc}$ collects field and method types from $\spc$ and from its parents}

  \end{mathpar}

  \bigskip
  \begin{minipage}[t]{0.6\columnwidth}
    \fbox{$\sptenv \wtsub \spteval \spcompat \spteval$}
    \begin{mathpar}
      \inferrule*{
        \sptenv \wtsub \spteval_0 \spsubt \spteval_1
      }{
        \sptenv \wtsub \spteval_0 \spcompat \spteval_1
      }

      \inferrule*{
      }{
        \sptenv \wtsub \spteval_0 \spcompat \sptdyn
      }
  \end{mathpar}

  \end{minipage}\begin{minipage}[t]{0.4\columnwidth}
    \fbox{$\spteval \spmatr \spteval$}
    \begin{mathpar}
      \inferrule*[right={$\spteval_0 \neq \sptdyn$}]{
      }{
        \sptdyn \spmatr \spteval_0
      }
    \end{mathpar}
  \end{minipage}

  \bigskip
  \fbox{$\sptenv \wtsub \spteval \spsubteq \spteval$}~reflexive, transitive closure of the following $\spsubt$ relation:
  \begin{mathpar}
    \inferrule*{
    }{
      \sptenv \wtsub \spteval_0 \spsubt \sptobject
    }

    \inferrule*{
    }{
      \sptenv \wtsub \sptbool \spsubt \sptint
    }

    \inferrule*{
      \sptenv \wtsub \spteval_0 \spsubt \spteval_1
    }{
      \sptenv \wtsub \sptoptional{\spteval_0} \spsubt \sptoptional{\spteval_1}
    }

    \inferrule*{
      \sptenv \wtsub \spteval_0 \spsubt \spteval_1
    }{
      \sptenv \wtsub \spteval_0 \spsubt \sptoptional{\spteval_1}
    }

    \inferrule*{
      \sptclassdef{\sptclass_0}{\sptclass_1}{\ldots}{\ldots} \in \sptenv
    }{
      \sptenv \wtsub \sptclass_0 \spsubt \sptclass_1
    }

%    \inferrule*{
%    }{
%      \spteval_0 \spsubt \spteval_0
%    }
%
%    \inferrule*{
%      \spteval_0 \spsubt \spteval_1
%      \\
%      \spteval_1 \spsubt \spteval_2
%    }{
%      \spteval_0 \spsubt \spteval_2
%    }
  \end{mathpar}

  \caption{Expression typing, consistent subtyping, and materialization}
  \label{f:eval-types}
\end{figure}


To show where \SP{} needs to insert casts, we present a selection of typing rules.
Recall that a program declares variables, functions, and classes~(\cref{f:surface-types}).
These declarations fill a type environment ($\sptenv$).
Relative to the current environment, the expressions within each statement must
satisfy the typing judgment in~\cref{f:eval-types}.

By contrast to our Redex model, the typing judgment in~\cref{f:eval-types} is
not syntax directed.
The rules do not describe a typechecking algorithm.
Instead these rules declare various conclusions and their preconditions to
illustrate how \SP{} reasons about types---dynamic or otherwise.

The first two rules are for function application.
A typed function may be applied to an argument that matches its domain type,
in which case it computes a value that matches its codomain type.
A dynamically-typed variable may be applied to any input (that matches the top
type $\sptobject$) to yield a dynamically-typed result.
The next rules illustrate writes to Python dictionaries and checked dictionaries.
A shallow Python dict may be updated with any kind of key and value.
By contrast, a concrete checked dict requires keys and values that match its type parameters.
Next we have two rules for classes:
\trule{F-Set} says that writes to a class field must match the declared field
type; and \trule{M-App} shows that typed methods impose similar constraints as
typed functions.

The final two rules, \trule{C-Sub} and \trule{Matr}, depend on auxiliary judgments
for consistent subtyping ($\spcompat$) and materialization ($\spmatr$).
Consistent subtyping relates type $\spteval_0$ to type $\spteval_1$ if they are
related by static subtyping ($\spsubteq$) or if
$\spteval_1$ is the dynamic type.
Materialization relates the dynamic type to any non-dynamic type.
Occurrences of the \trule{Matr} rule are downcasts that require a
run-time check.

The name \emph{materialization} and the rule \trule{Matr} are inspired by prior
work~\cite{clps-popl-2019}.
By itself, the judgment is merely an upside-down type precision relation~\cite{g-icfp-2013,sv-dls-2008}.
Combined with the typing rule, however, materialization is a concise way to find
where a well-typed program needs casts to ensure soundness.
For example, suppose that $\spf$ is a function from integers to the dynamic
type and that $\spx$ is a variable with the dynamic type.
An application $\spapp{\spf}{\spx}$ can satisfy the type $\sptstr$ using two materializations:

\medskip
\begin{mathpar}
  \inferrule*[left=Matr]{
    \inferrule*{
      \inferrule*[left=Matr]{
        \sptenv \wtexpr \spx : \sptdyn
        \\
        \sptdyn \spmatr \sptint
      }{
        \sptenv \wtexpr \spx : \sptint
      }
    }{
      \sptenv \wtexpr \spapp{\spf}{\spx} : \sptdyn
    }
    \quad
    \sptdyn \spmatr \sptstr
  }{
    \sptenv \wtexpr \spapp{\spf}{\spx} : \sptstr
  }

  \mbox{Where $\sptenv = \sptvardef{\spx}{\sptdyn}, \sptfundef{\spf}{\sptint}{\sptdyn}$}
\end{mathpar}
\medskip
Consequently, this derivation calls for two casts at the \trule{Matr} applications.

We end here, with materialization, rather than present a semantics
for the formalization.
After all, the main benefit of a full formal semantics is to validate
the behavior of complex expressions---and this job is
better left to the Redex mechanization, which at any rate supports a much
larger subset of Python~(\cref{s:surface}).

Furthermore, our casts-via-materialization rule assumes that the
type checker has access to the whole program.
In practice, \SP{} approximates our ideal casts in two steps
to support interactions with unanalyzed Python modules.
First, \SP{} inserts casts in an overapproximate way:
\begin{itemize}
  \item
    typed functions and methods begin by checking all their arguments;
  \item
    typed classes check all field writes;
  \item
    concrete-typed dictionaries check all key and value writes; and
  \item
    typed code checks the results of function calls, references, and other elimination forms
    whenever there is a materialization from the dynamic type.
\end{itemize}
Second, the type-directed optimizer skips casts wherever a well-typed context
interacts with a typed value.
\Cref{s:impl} explains the optimizer in more detail.


%\subsubsection{Program Typing, Inheritance Restriction}
%
%\Cref{f:program-typing} presents the typing judgment for programs,
%which folds over a sequence of statements to build a type environment.
%The typing rules for variables and functions are standard.
%We draw attention to the rule for classes, in particular the override condition.
%A class may override fields and methods from its parent;
%if it does, the overriding definition must be a \emph{static subtype} of the original.
%For example, a method that returns an integer can be overridden by a method that returns
%a boolean, but it cannot be overriden by a method that returns the dynamic type.
%This restriction lets the \SP{} compiler avoid some run-time checks; it can assume
%that static methods return reliable values instead of having to insert checks for subtypes.
%
%\begin{figure}[t]
%  \fbox{$\sptenv \wtprog \spprog \dashv \sptenv$}
%  \begin{mathpar}
%    \inferrule*{
%      \sptenv_0 \wtprog \spstmt_0 \dashv \sptenv_1
%      \\\\
%      \sptenv_1 \wtprog \spprog_0 \dashv \sptenv_2
%    }{
%      \sptenv_0 \wtprog \spstmt_0, \spprog_0 \dashv \sptenv_2
%    }
%
%    \inferrule*{
%      \sptenv_0 \wtexpr \spexpr_0 : \spteval_0
%    }{
%      \sptenv_0 \wtprog \spvardef{\spx_0}{\spteval_0}{\spexpr_0} \dashv \sptvardef{\spx_0}{\spteval_0},~\sptenv_0
%    }
%
%    \inferrule*{
%      \sptenv_1 = \sptfundef{\spf_0}{\spteval_0}{\spteval_1},~\sptenv_0
%      \\\\
%      \sptenv_1 \wtexpr \spexpr_0 : \spteval_1
%    }{
%      \sptenv_0 \wtprog \spfundef{\spf_0}{\spann{\spx_0}{\spteval_0}}{\spteval_1}{\spexpr_0} \dashv \sptenv_1
%    }
%
%    \inferrule*{
%      \sptenv_1 = \sptclassdef{\spc_0}{\spc_1}{\sptvardef{\spx_0}{\spteval_0}}{\sptfundef{\spf_1}{\spc_0, \spteval_1}{\spteval_2}}, \sptenv_0
%      \\\\
%      %% TODO are field overrides exact?
%      \mbox{if $\spann{\spx_0}{\spteval_0'} \in \spenvapp{\sptenv_1}{\spc_1}$ then $\spteval_0 = \spteval_0'$}
%      \\\\
%      %% TODO can T override dyn?
%      \mbox{if $\sptfundef{\spf_1}{\spc_2, \spteval_1'}{\spteval_2'} \in \spenvapp{\sptenv_1}{\spc_1}$ then $\spteval_1' \spsubteq \spteval_1$ and $\spteval_2 \spsubteq \spteval_2'$}
%      \\\\
%      \sptenv_1 \wtexpr \spvardef{\spexpr_0} : \spteval_0
%      \\
%      \spann{\spself}{\spc_0}, \spann{\spx_1}{\spteval_1}, \sptenv_1 \wtexpr \spexpr_1 : \spteval_2
%    }{
%      \sptenv_0 \wtprog \spclassdef{\spc_0}{\spc_1}{\spvardef{\spx_0}{\spteval_0}{\spexpr_0}}{\spfundef{\spf_1}{\spself, \spann{\spx_1}{\spteval_1}}{\spteval_2}{\spexpr_1}} \dashv \sptenv_1
%    }
%  \end{mathpar}
%
%  \caption{Program typing}
%  \label{f:program-typing}
%\end{figure}


\subsection{Type Boundary Soundness}
\label{s:ts}

The soundness of our model is easily seen by reduction to Nom, a gradual
language that allows fine-grained interactions with untyped code and comes with
a detailed proof of soundness~\cite{mt-oopsla-2017,mt-oopsla-2021}.
Given a program written in the syntax of \cref{f:surface-types}, one can derive
a Nom program with the same type boundaries by replacing every checked
dict instance with a fresh class type.
Because distinct checked dict types are incompatible, the
original boundaries impose the same constraints as the translated Nom ones.

For interested readers, we offer the following direct argument as well.
Boundaries to less-typed code would be \emph{unsound} if an untyped
(or partially-untyped) value could enter a typed context without
a validating check.
The question is thus whether all boundaries are properly guarded.
Our answer has two parts.
First, observe that every evaluation type comes with a decidable
cast~(\cref{t:cast}).
For basic types such as strings, a tag check clearly suffices.
For parameterized types such as $\sptchkdict{\spteval_0}{\spteval_1}$,
the exact-match semantics is also decidable (and also implemented with
tag checks).
Second, observe that boundaries can only relate the dynamic type to an
evaluation type; there are no significant boundaries that relate a partially
static type to an evaluation type.
This is due to the exact-match semantics for checked types and the fact that
$\sptoptional{\sptdyn}$ normalizes to the dynamic type.
Therefore casts at static occurrences of the materialization rule are
enough to protect all boundaries.


\section{Scaling to Python}
\label{s:impl}

The model intentionally does not cover all of Python.
Some aspects of \SP{} are left out because they are straightforward to
handle.
These include primitive types~(\cref{s:c-types}), module-level variables,
and the boundary to untyped Python~(\cref{s:py-boundary}).
%% Aspects of \SP{} that do not pertain to soundness, such as the quality of error messages,
%% are also omitted.
Other Python features are left out because their \SP{} semantics is identical
to standard Python~(\cref{s:py-boundary}).
Our model also does not cover the optimizer~(\cref{s:optimize}) because
a formal account of type-directed optimization would require substantial additional
components; namely, a model of the \SP{} bytecode and a faithful rendering of its
transformations.


\subsection{Interactions with Open-World Python Code}
\label{s:py-boundary}

Although \SP{} is technically a new language, it is designed for gradual adoption.
Python programmers should be able to add types module-by-module to an
existing codebase.
Consequently, \SP{} supports interactions with Python modules in the only viable
way: by letting the Python code run with zero static constraints and minimal
dynamic constraints.

%% TODO FILL only 2 cases? revisit after adding "changes to python" section in 2
There are two cases in which \SP{} types impose dynamic constraints on untyped Python code.
First, concrete types (e.g., $\sptrawchkdict$)
reject inputs that were not created via a checked constructor.
If a client of a Python module decides to impose concrete types, the Python module
may need to change.
Second, typed modules prevent updates to module-level variables.

Interactions with {open-world}~\cite{vss-popl-2017} Python code pose a minor
threat to soundness because
typed functions (and methods) cannot assume that their arguments are
well-typed.
All arguments sent from typed contexts get validated either statically or
via materialization casts, but arguments from untyped contexts are unchecked.
For this reason, \SP{} compiles every typed function to check its inputs.
The optimizer can bypass these checks for typed-to-typed calls~(\cref{s:optimize}).
Similarly, concrete types such as $\sptrawchkdict$ must check writes that
come from untyped contexts.
%% 2022-01-09 worth mentioning primitive types? they get cast like anything else (they don't materiazize)
%% 2022-01-13 don't think so --- leave prims alone, we don't want to frighten reviews with unsoundness


\subsection{Dynamic Python Features}
\label{s:dynamic-python}

%% TODO 2022-01-09 be careful here ... the model may indeed handle module vars
%%  the thing is, they're untyped and (maybe) not tested

\SP{} does not ascribe types to the following Python features.
These are not covered in the model because the implementation simply
assigns the dynamic type and lets the runtime treat them as untyped
Python code.
For each, the dynamic type is a reasonable choice;
accurate static types would be burdensome to maintain.


\paragraph{First-Class Classes}

\SP{} does not attempt to type first-class classes: partly because they have
yet to appear in performance-critical code in the Instagram web server
and partly because they do not fit well with a nominal type system.
The straightforward but restrictive approach for first-class nominal class types
is to force code that uses a first-class class to expect subtypes of a
particular named static class.
Flatt et al.~\cite{fkf-popl-1998} propose a more flexible approach, %(specifically for mixins)
but it requires a second layer of \emph{interface types} atop the nominal hierarchy.
MonNom~\cite{mt-oopsla-2021} uses interfaces in a similar way to accommodate
structural objects.

On a related note, \SP{} has no support for first-class functions or for
structural types as defined by PEP~544~\cite{pep544}.
MonNom interfaces may be a promising way to support these types.

% in particular , Python's multiple inheritance would complicate the mixin story.
% Classic mixins rely on structural types~\cite{bc-oopsla-1990}.

%Notes on first-class classes being untyped:
%- mixin-based code tends to structrural types
%- nominality overly restrictive, get forced into <: hierarchy
%- but don't have NO IDEA for wat to do
%  + classes and mixins paper shows one idea,
%    ... need complicated mixin study??? (many sub objects)
%    built new type system to be mixin-aware, layered atop java
%    unsure if such extension works for SP
%    furthermore, class-and-mixins depends on Java single inheritance
%  + bracha cook mixin pattern
%  + tate muehlboeck structural object + interfaces


\paragraph{Multiple Inheritance}

%% https://docs.python.org/3/reference/datamodel.html
%% TODO check Dino email, confirm with C&D that layout conflict is unrelated to vtables

Python allows classes to inherit from a list of parents.
\SP{} accommodates this behavior provided that at most one parent
is typed (or uses \code{\_\_slots\_\_} to specify its layout; plain Python has the same restriction).

\SP{} does not leverage types for classes with multiple parents
because of vtable layout issues.
Efficient vtable dispatch requires that each method name
resides in a fixed slot across all subclasses.
With multiple inheritance, however, parents may disagree about which method
lives in each slot.
Similarly, \SP{} cannot optimize instance variable reads because their slots
may have a conflict among parents.
%% possible improvement: invoke methods on the first parent statically and on
%% other parents dynamically and forbid those other paarents from having instance
%% attributes.

%% So in normal Python you will get a layout error anytime you try to inherit from
%% more than one class defining _slots_. Any SP class with instance attributes
%% implicitly defines _slots_. So it depends how you interpret "where Python
%% wouldn't." If you consider SP classes to be slots-defining (because we emit
%% bytecode such that they are, even though they don't explicitly in the source
%% code), then our behavior is just the normal Python behavior.
%% 
%% There is one edge case where we are more strict, which is a SP class with no
%% instance attributes, only methods. In this case it will not have _slots_ and
%% Python would allow it to multiply-inherit with something else that does have
%% _slots_. But because we also have the vtables problem (which normal Python does
%% not), we can't ever allow multiple inheritance from two SP classes, even if one
%% or both of them has no instance slots, and even if that multiple inheritance
%% occurs in a nonstatic module (the nonstatic subclass will still have a vtable,
%% because we may statically invoke methods on it, since it is substitutable for
%% its static parents, and that vtable will still be impossible to construct
%% correctly.) We implement this restriction in a kind of lazy way by just
%% piggybacking on Python's existing layout incompatibility check, and adding in a
%% clause that forces every SP class to always be considered as slots-having, even
%% if it doesn't actually have any instance variable slots. So in this edge case
%% only, we will error where Python wouldn't.
%% 
%% I summarized all of this in as few words as I could with "at most one parent is
%% defined in typed code or uses _slots_ to specify its members," which is what
%% the above boils down to.
%% 
%% suppose technically we are slightly more restrictive than we would have to be:
%% we currently forbid, but technically could support a case of multiple
%% inheritance where one parent is an SP class with no instance attributes, and
%% the other parent is a nonstatic class. But this seems not very useful to
%% support, and just makes the rules even more complicated.


\paragraph{Dynamic Execution}

%% 2022-01-13: __setattr__ is along the same lines
%%Carl: Removing the use of super-dynamic Python features like
%%\code{\_\_setattr\_\_} that we don’t support. So far in the cases where
%%we’ve done this it hasn’t required a major rewrite.

Calls to \code{eval} and \code{exec} have the dynamic type.
%The type system makes no finer prediction about what kind of value such calls will
%return.
%
Studies of JavaScript and R have shown that many uses of dynamic execution
can be removed through simple adjustments~\cite{rhbv-ecoop-2011,gdkkv-oopsla-2021,mrmv-esop-2012}.
Assuming these findings carry over to Python, similar adjustments would be
preferable over attempting to type \code{eval}.


%% 2022-01-13 covered pretty well earlier
%%\paragraph{Module-Level Variables}
%%
%%Any Python module can read and write to the module-level variables of another module.
%%When compiling a module, \SP{} therefore assumes that its module-level variables
%%may be modified by untyped code and assigns the dynamic type.
%%
%%Strict modules as an alternative to the Python
%%semantics.\footnote{https://instagram-engineering.com/python-at-scale-strict-modules-c0bb9245c834}
%%If a programmer chooses to declare a module as strict, then its module-level varibles
%%are immutable and thus typeable.
%%
%%Another potential solution is for the Cinder runtime to check that writes to module-level variables
%%preserve their types.
%%There are two downsides to this idea:
%%it will add some performance overhead,
%%and it will force developers to rewrite untyped code in order to fix any errors that arise.
%%%% Because of the latter concern, type dynamic is a very reasonable default.


\subsection{Bytecode and Optimizations}
\label{s:optimize}

\begin{table}
  \caption{Representative Cinder bytecode instructions}
  \label{t:bytecode}

  \centering
  \begin{tabular}{lll}
    {Instruction} & {Purpose} & {Description} \\\hline
    \bcinst{CAST} & Soundness & Assert that a value matches a type \\
    \bcinst{CHECK\_ARGS} & Soundness & Cast all inputs to a function \\[1ex]

    \bcinst{BUILD\_CHECKED\_MAP} & Constructor & Make a checked dictionary \\
    %\bcinst{BUILD\_CHECKED\_LIST} & Constructor & Make a checked list \\
    \bcinst{TP\_ALLOC} & Constructor & Make a \SP{} object \\[1ex]

    \bcinst{INVOKE\_FUNCTION} & Optimization & Execute a direct function call \\
    \bcinst{INVOKE\_METHOD} & Optimization & Execute a method via the object's vtable \\
    \bcinst{LOAD\_FIELD} & Optimization & Read an object field \\
    \bcinst{STORE\_FIELD} & Optimization & Write to an object field \\
    \bcinst{FAST\_LEN} & Optimization & Get the length of a built-in value \\
    \bcinst{REFINE\_TYPE} & Optimization & Type declaration for the JIT \\

%%% Other bytecodes ... either from Carl's talk or the SP repo
%CONVERT_PRIMITIVE
%INT_LOAD_CONST_OLD
%JUMP_IF_NONZERO_OR_POP
%JUMP_IF_ZERO_OR_POP
%LIST_DEL
%LOAD_ITERABLE_ARG
%LOAD_LOCAL
%LOAD_MAPPING_ARG
%POP_JUMP_IF_NONZERO
%POP_JUMP_IF_ZERO
%PRIMITIVE_BINARY_OP
%PRIMITIVE_BOX
%PRIMITIVE_COMPARE_OP
%PRIMITIVE_LOAD_CONST
%PRIMITIVE_UNARY_OP
%PRIMITIVE_UNBOX
%RETURN_PRIMITIVE
%SEQUENCE_GET
%SEQUENCE_REPEAT
%SEQUENCE_SET
%STORE_LOCAL

  \end{tabular}
\end{table}

\SP{} can generate efficient code because it targets
the Cinder runtime, which has bytecode instructions to check types at run-time, to
construct \SP{} data structures, and to perform type-directed actions.
\Cref{t:bytecode} lists a few representative bytecode instructions.
The two instructions that express run-time checks are \bcinst{CAST}
and \bcinst{CHECK\_ARGS}.
The former checks a value against a type.
The latter is for functions and methods; it checks all inputs to a function
against their declared types.
The instructions \bcinst{BUILD\_CHECKED\_MAP} and \bcinst{TP\_ALLOC}
allocate \SP{}-specific data structures.
The first creates a checked dictionary; the second allocates an uninitialized instance
of an object~(\cref{s:vtable}).
Cinder comes with a similar instruction to build checked lists.
The remaining instructions are for optimization.
Both \bcinst{INVOKE\_FUNCTION} and \bcinst{INVOKE\_METHOD} are alternatives
to Python's dynamic call dispatch.
The former uses the classloader to quickly find the address of a function;
the latter uses a vtable lookup to find a method.
The load and store instructions for fields improve upon Python's generic attribute lookup.
In the JIT, these instructions can be further optimized to a single assembly instruction.
Lastly, the \bcinst{REFINE\_TYPE} instruction tells the JIT about the type of a local value
when it is not clear from the context.


\paragraph{Reducing Casts}

In addition to upgrading bytecode instructions to optimized ones, \SP{} takes care
to minimize the type casts that it executes at run-time.
In other words, it takes care to slow code down as little as possible.

Part of this goal is met by inserting casts only in positions where the dynamic
type flows into a static type.
The materialize rule in the model illustrates this policy, which ensures that
well-typed writes to classes and to concrete dictionaries have no cost.
Typed functions, however, are compiled with a \bcinst{CHECK\_ARGS} instruction
that defensively casts all arguments.
By convention, this instruction always appears on the first line of a typed
function body.
The optimizer uses this convention to skip argument checks by jumping past them
when it is safe to do so; that is, whenever a typed function or method gets
called in a typed context.


\section{Production Experience}
\label{s:eval}


%% 2022-01-04 TODO have no evidence that "any module" typing works out ... team
%% tried hard to get network effects ... "We explicitly aimed for converting
%% related modules like this to improve network effects and get better
%% performance."

%% TODO 2021-12-29 Carl
%% We definitely did find some regressions (especially when converting Cython, but
%% also one or two in non-Cython conversions.) But we were never satisfied with a
%% regression, and we were always able to apply the JIT or improve SP
%% optimizations or adjust the code to turn the regression neutral or positive.
%% (In fact I think all? the non-Cython regressions I can recall boiled down to
%% “we forgot to enable the JIT for an important function” and just rectifying
%% that took care of it.)

The Instagram web server has been using \SP{} code to serve requests in
production since April 2021.
Overall, the results are very encouraging.
Instagram's internal profiling tools attribute a \CPUchange{} improvement
in CPU efficiency~(\cref{s:what-is-cpu}) to \SP{} conversions~(\cref{s:migration-path}).
This is a big improvement at Instagram scale.

Developers have converted over 500 modules to \SP{} thus far.
Despite some initial regressions, none of the converted modules
ran slower after small rewrites~(\cref{s:code-changes})
and/or enabling the JIT on certain functions.
Only 9 modules use concrete types and only 12 modules use primitive types,
but these features delivered critical performance in key modules.

%% all prims are int32 and int64 and cbool
%% 541 modules out of ~45000 total
%% 873 modules use JIT for some functions (about to increase by OOM)


\subsection{How to Interpret the CPU Efficiency Result}
\label{s:what-is-cpu}

The Instagram web server code runs on a large number of host machines, each of
which continuously handles requests from a common pool.
Improvements to the server codebase should make these machines more efficient
as they handle arbitrary requests.

To measure the effect of a code change on these hosts, an internal profiling tool selects
two representative sets of machines to run as experimental and control groups.
The experimental group gets the latest version of the server code;
the control group gets the previous version.
Next, the profiling tool slowly increases the number of requests that these
machines receive until each is running at maximum load.
Once the machines are fully allocated, the tool measures how many requests
per second each group is able to handle.
If the experimental group can serve more requests per second, then the change
is a success.
Assuming that the experimental and control groups have equivalent hardware
capabilities, and assuming that both groups receive a representative sample
of requests, then the results of an experiment (increase/decrease in CPU efficiency)
should predict actual performance.

The one caveat with this CPU efficiency measure is that it is pinned
to a specific point in time.
Changes in product code and in client behavior can change the size
of a typical web server request, which changes the number
of requests that a server can handle per second.
Taking the sum of several CPU efficiency changes over a long
timespan (as we have done) is therefore only a rough
measure of their net effect.
%% In particular, although we report a \CPUchange{} net improvement
%% in web server CPU efficiency due to \SP{}, removing \SP{} and
%% taking a new measurement may yield a slightly different degredation
%% than \CPUchange{}.


\subsection{Migration Path}
\label{s:migration-path}

\SP{} first entered the web server codebase as a replacement for a
few critical Cython modules.
Cython improved performance by compiling these modules to C,
but its partial adoption had led to an awkward workflow with a
few compiled modules spread across an interpreted codebase.
Replacing these modules with \SP{} lets programmers return to a conventional
Python workflow.
Furthermore, \SP{} types combined with Cinder resulted in a
0.7\% improvement in web server CPU efficiency.
Primitive types~(\cref{s:c-types}) were essential for matching Cython.

Later \SP{} migrations have been directed by profiling to find frequently-executed code.
During the first half of 2021, the \SP{} team identified hot
modules and proposed types plus small code changes to the maintainers of these modules~(\cref{s:code-changes}).
The accepted proposals resulted in a 1\% improvement in CPU efficiency.
During the second half of 2021, the \SP{} team applied the same
process at a larger scale.
They also modified a code-generating module to output \SP{} code;
this one change brought over four hundred generated modules into the typed fold.
Overall, the second-half changes added 2\% to the number of requests
that servers could handle per second at maximum CPU load.

As of December 2021, the \SP{} team has converted 541 modules.
Most of these came from the code-generating tool (417); the rest
are from hand conversions (124).
These modules frequently interact with untyped modules in the codebase.
According to an analysis of identifiers that cross module-dependence
boundaries, over 30,000 identifiers go between typed and untyped code.
Two-thirds of these crossings are exports from \SP{} modules
to untyped modules; in other words, typed identifiers are widely used
throughout the web server.
%% data in src/static-python-imports-analysis
%% TODO gotta compare to TR etc.

% In a typical week, the team adds types to four additional modules.
% These modules have seen speedups ranging from 0\% to 25\% depending
% on their nature.
% Regressions have not been an issue.
% The only cause for alarm is when a module gets only a small improvement
% without code changes.


\subsection{Analyzing Code Changes}
\label{s:code-changes}

Because the \SP{} team has been changing code as well as types during its
migrations, the question arises as to whether the changes themselves
are responsible for more speedups than the types and gradual soundness.
%% Carl: "I reviewed the ~30 conversion diffs"
We therefore reviewed conversion diffs that significantly improved
performance (\nnum{30}) to assess the extent of the code changes.
There are nine common kinds of changes: five that satisfy the type checker
and four that aim to improve performance.

%% one significant change: replaced `contextlib.ContextDecorator` with `__static__.ContextDecorator`
%%  removes an extra call layer
%%  helps with (timing [ context managers / decorators ])

\subsubsection{Code Changes for the Type Checker}
\label{s:behavior-changes}

\begin{enumerate}
  \item
    \ipara{Fix type errors due to mock wrappers}
    When test code uses a mock wrapper, it may change the
    return type of a function because wrappers return a \code{MagicMock} object
    by default.
    The fix is to specify a return value.
    %% Carl: This is an inherent and desired
    %% incompatibility; next half we plan to work on improving the Python mock
    %% framework to automatically create mocks of the correct types.

  \item
    \ipara{Change mocked function to expect positional arguments}
    \SP{} currently rewrites the bytecode for all typed-to-typed function calls to use positional arguments.
    Mock-wrapped functions therefore need to use positional arguments.
    %% Carl: we plan to fix this incompatibility by falling back to the original call
    %% arguments when we detect the target function has been patched.

  \item
    \ipara{Organize class and instance attributes}
    Whereas Python allows class attributes to serve as default values for instance attributes,
    \SP{} does not.
    %% Programmers have to decide where an attribute belongs so that \SP{} can optimize
    %% reads from instance attributes.
    %% Carl: (efficient via \code{LOAD\_FIELD})
    %% - related to github #37 ?

  \item
    \ipara{Move unsupported Python features}
    Code that uses the unsupported features listed in~\cref{s:not-python}
    must be changed or moved to an untyped module.
    %% So far, these changes have not required major rewrites.

  \item
    \ipara{Refine some Pyre annotations}
    Well-typed Pyre code is not always well-typed \SP{} code.
    \Cref{a:pyre-limits} lists the main friction points.

\end{enumerate}


\subsubsection{Code Changes for Performance}

\begin{enumerate}
  \item
    \ipara{Use primitive types}
    Integers and booleans in hot code paths run fastest with
    primitive types~(\cref{s:c-types}).
    %%As noted above, primitives were required for \SP{} to match the performance of
    %%Cython.
    Twelve modules currently use primitives.
    %% - bg: focus on block, convert, no problem?

  \item
    \ipara{Use concrete types}
    Three modules currently use a concrete dictionary type
    (of the form $\sptchkdict{K}{V}$)
    and six modules use a checked list type.
    %% - bg: how occasional? worried about power of the "gradual soundness" claim

  \item
    \ipara{Change functions to accept positional arguments}
    Functions calls that use only keyword arguments are not yet optimized by \SP{}.
    %% Rewriting these functions and their call-sites to use positional arguments improves performance.

    %% Carl: We plan to support this next half.

  \item
    \ipara{Convert \code{@classmethod} to \code{@staticmethod}}
    Static methods can get invoked directly as functions, bypassing the class vtable.
    Twenty-four modules use static methods (many of these pre-date \SP{}).
    %% 2022-01-12 Carl:
    %% I don’t think it was a super common change, but there were some cases where we
    %% changed a @classmethod that didn’t actually use its cls argument to a
    %% @staticmethod. It looks like we have 24 static modules using a
    %% @staticmethod somewhere in the module, but I think in the majority of those
    %% there just happened to be some pre-existing staticmethods (they aren’t
    %% uncommon in our codebase); I think a minority were intentional
    %% optimizations on our part. And the pre-existing ones may or may not be
    %% meaningfully improving performance. So I’m not sure how valuable it is to
    %% report this metric.


\end{enumerate}

\subsubsection{Conclusion}

In general, the patches responsible for performance gains contain minor
code changes and 
affect test code rather than production code.
We conclude that the addition of \SP{} types to a codebase can significantly
improve performance without major refactoring.


%% More Migration Notes

%% 2022-01-12 Carl:
%% We did have some cases where strict semantics caused a problem, but in the
%% majority of conversions it didn’t (or if it did initially, the fix was
%% trivial.) There was at least one case where I would have converted more code to
%% static but couldn’t easily do so because of strictness issues with a larger
%% framework reliant on metaclasses, and so I instead extracted the hottest code
%% paths into a new strict/static module and called them. I have a plan to go back
%% to this and improve the framework itself to be strict so more code can be
%% converted. I know of at least one other case where we did modify a framework
%% that relied on side-effecting decorators so that it could be used without the
%% side effects.

% 2022-01-24 bg: hidden from submission to save space
%\subsection{Developer Experience}
%
%For the most part, programmers have been happy adopting \SP{}.
%The transition from Pyre to \SP{} has delivered performance benefits
%that outweigh the cost of code changes.
%Even the use of strict modules has been acceptable; only two modules
%required non-trivial rewrites to avoid mutation.
%
%One significant pushback dealt with multiple inheritance.
%The \SP{} team proposed a rewrite from multiple inheritance to single inheritance,
%but the code maintainers preferred to keep the design as-is instead of enabling
%type-driven method dispatch optimizations.


\subsection{Microbenchmarks}
\label{s:microbenchmarks}
%% https://github.com/facebookincubator/cinder/tree/cinder/3.8/Tools/benchmarks
%% https://docs.google.com/spreadsheets/d/1IO25bL_8LwDqx3ciQTj9dMetsFnZFdiQIyjFvgYWIpI/edit?usp=sharing

\Cref{t:microbenchmark} compares \SP{} to Python on public microbenchmarks.
These microbenchmarks are admittedly small and quite different than typical application code,
but they provide an auditable and reproducible way to measure performance.

Each row presents two ratios.
The first compares a simply-typed version of the benchmark to untyped Python code.
The second compares a hand-refined benchmark to untyped Python.
The refinements are modest, mostly adding primitive types to hot loops.
%% they did not require significant expertise to create.
All runtimes for this table came from a 64-bit CentOS Linux system using a build
of \SP{} and Cinder compiled with default settings.
Each benchmark version ran 11 times total: once to warm up the bytecode caches
(\code{.pyc} files) and 10 times to compute an average runtime (using the
\code{time} utility).
Every benchmark version invoked the JIT on the same set of functions.
\Cref{a:microbench-detail} presents further details.

\begin{table}[t]
  \caption{Microbenchmark performance ratios}
  %% FILL don't forget to update \numbenchmark{}
  \label{t:microbenchmark}

  \centering
  \begin{tabular}{lrr}
    Name               & Typed / Python & Refined / Python \\\midrule
    \bmname{deltablue} &         0.59 &           0.30 \\
    \bmname{fannkuch}  &         1.03 &           0.46 \\
    \bmname{nbody}     &         1.09 &           0.24 \\
    \bmname{richards}  &         0.53 &           0.22
    %% fannkuch typedbasic2 = 0.90
  \end{tabular}

\end{table}

Almost all ratios are less than one, showing that \SP{} achieves a speedup over
Python (typed avg.~$0.8$, refined avg.~$0.3$).
The exceptions are the unrefined versions of \bmname{fannkuch} and
\bmname{nbody}, which is because these benchmarks focus
on number-crunching and \SP{} optimizes arithmetic only for primitive types.
Changing a few types in \bmname{fannkuch} to primitives
gives an immediate speedup (ratio = $0.90$).


\paragraph{Threat to Validity}

\Cref{t:microbenchmark} does not analyze benchmark configurations that mix
typed and untyped code. %% ---only fully-typed and untyped configurations.
A study of mixed configurations (e.g.,~\cite{gm-pepm-2018,gtnffvf-jfp-2019})
may reveal performance bugs that have not yet shown up in production.


\section{Related Work}
\label{s:related}

Concrete types originated in Thorn~\cite{bfnorsvw-oopsla-2009,wnlov-popl-2010}
and StrongScript~\cite{rzv-ecoop-2015}, two early languages that support typed
and untyped objects.
Thorn is a scripting language for the JVM; StrongScript is designed to
interoperate with JavaScript.
Concrete types came to \SP{} by way of Nom~\cite{mt-oopsla-2017} and its
successor MonNom~\cite{mt-oopsla-2021}.
The impressive performance figures in the original Nom paper inspired the \SP{}
team to adapt the method.
But whereas MonNom has been exploring designs for generic concrete types that
support (a flexible form of) the gradual guarantees~\cite{svcb-snapl-2015},
\SP{} opts for a simpler and more rigid design~(\cref{s:type-dynamic}).

With its mix of shallow and concrete types, \SP{} is one of a few gradual
language that lets programmers choose between multiple type-enforcement strategies.
Thorn and StrongScript offer a choice of concrete types and optional (unsound)
\emph{like} types.
Typed Racket supports deep (structural) types, shallow types, and optional
types~\cite{g-pldi-2022}.
By contrast to these others, \SP{} asks programmers to change their code
in way that reflects changes to the types~(\cref{s:checked-type}).
The changes impose a maintenance burden, but enable performance improvements.

Many aspects of our Redex model for \SP{} are adapted from the Full Monty core calculus
for Python~\cite{pmmwplck-oopsla-2013}.
In particular, the Full Monty paper reminded us that booleans are integers in Python;
\SP{} had overlooked this detail.
%% \footnote{Bug report: \url{https://github.com/facebookincubator/cinder/issues/46}}
%% 2022-01-14 TODO check full monty with KC

Production experience with \SP{} suggests that a method-based JIT can
eliminate the costs of shallow types.
This finding complements prior work.
In the context of shallow types,
the pypy tracing JIT improved Reticulated~\cite{vsc-dls-2019}
and the Graal partial-evaluation-based JIT improved Grace~\cite{rmhn-ecoop-2019,grmhn-vmil-2019}.
In the context of concrete types, the HiggsCheck VM improved an implementation of
SafeTypeScript~\cite{rsfbv-popl-2015,rat-oopsla-2017}.
And in the context of deep/guarded/structural types, the Pycket tracing JIT greatly reduced the costs of higher-order
checks in Typed Racket~\cite{bbst-oopsla-2017}.
% In sum, there is much evidence to suggest that gradual languages go well with
% JIT technology.

\SP{} is one of many type systems for Python.
Other optional type checkers include Pyre~\cite{pyre},
mypy~\cite{mypy}, PyType~\cite{pytype}, and Pyright~\cite{pyright}.
These checkers use types only for static analysis; optional types have no
effect (positive or negative) at run-time.
Another sound type system is Reticulated, which enforces structural types and
pioneered the transient semantics~\cite{vss-popl-2017,v-thesis-2019}.
A rigorous comparison of Reticulated and \SP{} is an important topic for future work.
We conjecture that adapting code to Reticulated requires fewer edits than \SP{}
but delivers slower performance.
Optimizing type systems for Python include Reticulated and mypyc~\cite{mypyc}.
The latter compiles Python source to C extension modules;
it does not offer a pure-Python developer experience.

Although sound gradual types are rare in industry, there are at
least three other languages that provide them.
Dart 2 is a nominally-typed language with a dynamic type that is similar
to a top (object) type but statically allows all method calls~\cite{dart-types}.
C\# has a dynamic type that delays type checking until runtime~\cite{bmt-ecoop-2010}.
\JSPP{} is typed JavaScript that allows untyped values
using a catch-all \code{external} type~\cite{jspp}.
None of these four implement gradual types as envisioned by researchers~\cite{st-sfp-2006,tf-dls-2006,mf-toplas-2009,gktff-sfp-2006};
instead, they offer a pragmatic solution to interoperability problems.
%% Programmers can manually cast external values to a more precise type.
%% The non-immediate casts make copies (how does that work for functions?!)
%% also, how does JS++ protect arrays sent to JS?


%% \section{Future Work}
%% \label{s:future}
%% 
%% %% Immediate future: fix adoption friction points
%% %% - mocks need manual return_value ... "next half we plan to work on improving the Python mock framework to automatically create mocks of the correct types."
%% %% - mock calls have to rewrite to expect positional args, not keywords "plan to avoid rewrite if fn is patched"
%% %% - functions have to expect positionals (can't do keyword only)
%% %% - @dataclass enum.Enum
%% %%
%% %% significant selection bias toward low-hanging fruit
%% %% model: metaclasses
%% %% engineering: JIT profiling, instead of hand-requested JIT list
%% 
%% Adapt SP ideas to a new setting, test performance takeaways.
%% 
%% Improve SP with generic types, structural types (lambda), \ldots.
%% 
%% Use confined GT to relax ChkDict ... let programmers decide whether
%% the type should reject or convert untyped data.
%% Maybe a ChkDict function could compile to an "overloaded" version with
%% fast and slow paths for ChkDict vs normal dict.
%% 
%% Build an automatic migration tool for Checked data.
%% 
%% Build a static analysis that hoists transient annotations to an early, shared point.
%% Take care to give quality error messages.
%% 
%% Optimize Python integer operations (in addition to the c types).
%% Hard because syntax like \code{a + b} may be the result of either \code{a.\_\_add\_\_(b)}
%% or \code{b.\_\_radd\_\_(a)} depending on runtime types and behavior.


\section{Lessons}
\label{s:conclusion}

%% 2021-12-30 Whence Nominal GT?
%%
%% KC: I think in general adding gradual typing to Java/C# (an also Python) is boring:
%% - MonNom allows you to cast between an arbitrary type and dyn.
%% - dyn is effectively Object
%% - Upcasts, casts from an arbitrary type to Object, are already in the language
%% - Downcasts, casts from Object to an arbitrary type is also possible, but must be explicit in the fully static language
%% So MonNom effectively gives programmers another way to name Object, with which downcasts can be automatically inserted by the compiler.
%% I think this sentence covers all the interesting stuff about MonNom.
%%
%% SK: This matches my intuition that gradual nominal is just not very
%% interesting. KC puts this on a technical footing with his very astute comment,
%% which to me can be summarized as "every nominal language that has Object is
%% already gradual" [leaving aside the boring non-Object base type distinction].
%%     So the entire remaining game is one of making the programming interface
%% as painless as possible. Some of that is just a trivial matter of
%% syntactic defaults; some of it is making the type-checker more forgiving
%% and replacing static checks with dynamic checks. There doesn't seem to
%% be much else left.

\SP{} is an ambitious undertaking.
Its developers are maintaining a fork of the Python 3.8 runtime,
a type checker,
an optimizing ahead-of-time compiler,
and a method-based JIT compiler.
The team has converted hundreds of untyped modules to \SP{}
and plans to convert thousands more in the future.
And yet, \SP{} is restricted to set attainable goals:
%% for its implementors and for adopters:
\begin{itemize}
  \item
    The syntax of sound static types is relatively small compared
    to \PEP{}.
    Higher-order types are absent.
    The focus is instead on first-order objects and standard data structures~(\cref{s:ts-context}).
  \item
    The type checker analyzes entire modules rather than individual statements,
    and although it supports a dynamic type, dynamic type-checked code
    does not have the same freedom as untyped code~(\cref{s:type-dynamic}).
  \item
    Shallow types accommodate Python code but offer coarse
    soundness guarantees.
    %% These types have been the top priority for \SP{}.
    Concrete types are deeply sound, but add nonlocal constraints~(\cref{s:checked-type}).
    %% \SP{} supports few concrete types at the moment and plans to add more.
  \item
    Gradual class hierarchies are permitted, but restricted to single inheritance.
    Static-to-static method overrides cannot reduce the precision of types~(\cref{s:inheritance}).
    %% ---a restriction
    %% that enables optimizations but breaks the static gradual guarantee.
  \item
    Progressive primitive types are available to maximize performance~(\cref{{s:c-types}}).
    %%These types are intentionally incompatible with the dynamic type to help programmers
    %%avoid casts.
\end{itemize}
%
Many of these restrictions go against the grain of the mainstream
research community, but they have been effective in practice.
The current \CPUchange{} increase in Instagram web server CPU efficiency is a huge improvement
and was obtained incrementally without any careful planning to avoid performance bottlenecks.
By contrast, bottlenecks are a major concern for expressive, structurally-typed
gradual languages~\cite{gtnffvf-jfp-2019,tfgnvf-popl-2016}.

In conclusion, the gradual soundness approach of \SP{} seems to be a promising
way to realize the vision of gradual typing.
Determining whether this conjecture holds more broadly
calls for research on two fronts.
First, \SP{} must be applied to additional projects in a module-at-a-time manner.
A systematic evaluation of a few small projects would be
especially useful to find performance bottlenecks.
Second, the \SP{} approach should be adapted to other languages.
An optionally-typed language such as TypeScript would be an ideal starting
point.
%% Gradual, progressive soundness may prove itself to be an effective general strategy for
%% gradual types in practice.


\acks{
  \SP{} and the Cinder JIT were developed by
  Max Bernstein,
  John Biesnecker,
  Jacob Bower,
  Sinan Cepel,
  Tiansi Hu,
  Orvid King,
  Vladimir Matveev,
  William Meehan,
  Carl Meyer,
  Matt Page,
  Aniket Panse,
  Brett Simmers,
  Andrei Talaba,
  Dino Viehland,
  and Shiyu Wang.
  The Brown authors contributed to the language by modeling it independently; they
  are not part of the \SP{} team.

  Special thanks to Twitter, through which the two teams
  (specifically, Meyer and Krishnamurthi) met.
  This work was partly supported by Meta and by the US National Science Foundation.
  Greenman is supported by NSF grant 2030859 to the CRA for the \href{https://cifellows2020.org}{CIFellows} project.
}

{\sloppy
\printbibliography
}

\appendix

\section{GitHub Issues}
\label{a:github-issues}

The 25 GitHub issues that arose from our work may be found at the following three links:

\begin{itemize}
  \item \shorturl{https://}{github.com/facebookincubator/cinder/issues/created\_by/LuKC1024} (N=20)
  \item \shorturl{https://}{github.com/facebookincubator/cinder/issues/created\_by/bennn} (N=4)
  \item \shorturl{https://}{github.com/facebookincubator/cinder/issues/64} (N=1)
  %% extra, can't remember if we inspired Dino or not: https://github.com/facebookincubator/cinder/issues/32
\end{itemize}


\section{On Migrations from Pyre to \SP{}}
\label{a:pyre-limits}

Although the Instagram web server has extensive type annotations that are
checked by Pyre,
these annotations are not always enough to satisfy \SP{}.
In fact, code may have latent bugs that Pyre did not catch;
we list common reasons for such bugs below:

\begin{itemize}
  \item
    Pyre misses some bugs because it does not monitor run-time interactions
    with untyped code.

  \item
    Similarly, some uses of the Pyre dynamic type (\code{Any}) end up
    raising errors when \SP{} monitors their interactions with precisely-typed code.

  \item
    Occurrences of the special comment \lstinline$# pyre-fixme$ disable Pyre checks.

  \item
    Some annotations intentionally lie to reduce the work of maintaining types.
    For example:

    \begin{itemize}
      \item
        The standard \code{typeshed} repo of Python type annotations declares
        that a weak reference has the dynamic type~\cite{typeshed}.
        This declaration is easy to use because it does not force programmers
        to handle the case where their weakly-held object has been collected,
        but it is also a potential source of bugs.
      \item
        Mutable class attributes are covariant in Pyre and other
        Python type checkers, including mypy~\cite{mypy}.
      \item
        The meta-type \code{Type} is unsoundly covariant as well in Pyre
        and others.
      \item
        Argument splatting, i.e., applying a function that expects positional
        arguments to a list (\code{f(*lst)}), is not statically checked
        either.
    \end{itemize}

\end{itemize}

None of these issues are problems with Pyre.
Quite naturally, it cannot find bugs in code that it does not analyze!
The takeaway is simply that optional type systems can miss a variety of issues.


\section{Skipped \SP{} Regression Tests}
\label{a:banned-test}

As of January 9 (commit \href{https://github.com/facebookincubator/cinder/commit/6d61575e28b2d1f31a197f883486c5f012de98c1}{6d61575}),
the \SP{} test suite contains \numSPtotal{} tests.
We use \numSPtest{} of these tests for the model.
The remaining \numSPdiff{} tests are skipped for the following reasons:

\begin{itemize}
  \item 269 test primitive types.
  \item 62 test optional and keyword functions.
  \item 26 test the classloader.
  \item 24 test coroutines (\code{async}).
  \item 23 test list and dictionary comprehensions.
  \item 15 use format strings.
  \item 12 test class and object slots.
  \item The others (\nnum{106}) test a long tail of other topics
    including byte strings, memory management, specific decorators
    (\code{@staticmethod}), and optimizations.

%% actual list
%  \item 269 use C_types
%  \item 62 use fancy_args
%  \item 26 use xxclassloader
%  \item 24 use async_methods
%  \item 23 use comprehension_syntax
%  \item 15 use format_strings
%  \item 12 use __slots__
%
%  \item 10 use advanced_occurrance_typing
%  \item 8 use @staticmethod
%  \item 7 use prod_assert
%  \item 7 use python_scope
%  \item 6 use __call__
%  \item 6 use reveal_type
%  \item 4 use @_donotcompile
%  \item 4 use memory_management
%  \item 3 use ...
%  \item 3 use __setattr__
%  \item 3 use immutable_variables
%  \item 3 use shadow_frame
%  \item 2 use TypeVar
%  \item 2 use decorator
%  \item 2 use float
%  \item 2 use nested_class
%  \item 2 use nonexpressible_tests
%  \item 2 use optimize=1
%  \item 2 use sys.modules
%  \item 1 use @property
%  \item 1 use NamedTuple
%  \item 1 use Protocol
%  \item 1 use bad_tests
%  \item 1 use byte_strings
%  \item 1 use code_flag
%  \item 1 use fancy_function_type
%  \item 1 use final_classes
%  \item 1 use slicing_syntax
%  \item 1 use sorted
%  \item 1 use try_except_redeclare
%  \item 1 use with_traceback
\end{itemize}

%%\begin{table}
%%  %% 2022-01-25: seems to raise more questions than it answers ... next question is what each filter really means.
%%  \caption{List of filter names and a count of skipped test that failed each filter}
%%  \label{t:left-out}
%%
%%  \begin{tabular}{lr}
%%    Filter & Tests Caught \\\hline
%%    C\_types & 269 \\
%%    fancy\_args & 77 \\
%%    format\_strings & 60 \\
%%    xxclassloader & 37 \\
%%    async\_methods & 25 \\
%%    comprehension\_syntax & 23 \\
%%    \_\_slots\_\_ & 15 \\
%%    immutable\_variables & 14 \\
%%    python\_scope & 14 \\
%%    float & 11 \\
%%    advanced\_occurrance\_typing & 10 \\
%%    @staticmethod & 8 \\
%%    reveal\_type & 8 \\
%%    \_\_call\_\_ & 7 \\
%%    prod\_assert & 7 \\
%%    ... & 5 \\
%%    nested\_class & 5 \\
%%    @\_donotcompile & 4 \\
%%    decorator & 4 \\
%%    memory\_management & 4 \\
%%    sys.modules & 4 \\
%%    TypeVar & 3 \\
%%    \_\_setattr\_\_ & 3 \\
%%    byte\_strings & 3 \\
%%    shadow\_frame & 3 \\
%%    sorted & 3 \\
%%    nonexpressible\_tests & 2 \\
%%    optimize=1 & 2 \\
%%    weakref & 2 \\
%%    @property & 1 \\
%%    NamedTuple & 1 \\
%%    Protocol & 1 \\
%%    bad\_tests & 1 \\
%%    code\_flag & 1 \\
%%    fancy\_function\_type & 1 \\
%%    final\_classes & 1 \\
%%    slicing\_syntax & 1 \\
%%    try\_except\_redeclare & 1 \\
%%    with\_traceback & 1
%%  \end{tabular}
%%\end{table}


\section{Fine-Grained Benchmark Data}
\label{a:microbench-detail}

\newcommand{\colname}[1]{\textbf{#1}}

The tables in this section present fine-grained data for the microbenchmark programs.
Each benchmark has several versions of its code that accomplish the same work:
the original untyped version (\colname{Orig}) and a typed version (\colname{T-Max}) converted for maximum
performance under Static Python and the JIT.
Some benchmarks have a basic typed version (\colname{T-Min}) that adds only
type annotations and necessary casts to the original code,
without using any more advanced Static Python
features such as primitive types or otherwise optimizing the code for Static
Python.
Fannkuch has a second basic version (\colname{T-Min-2}) that uses just a few primitive integers
where it can be done without significant changes to the code.

The other axes of our test matrix are \colname{SP} (whether the Static Python compiler is
used), \colname{JIT} (whether the Cinder JIT is enabled), and \colname{SF} (whether the JIT
shadow-frame mode is enabled, which reduces Python frame allocation costs.) The
matrix is not complete, since using the Static Python compiler on untyped code
has no noticeable effect, and shadow-frame mode is only relevant under the JIT.

Benchmarks are run on a 64-bit CentOS Linux system, using a build of Cinder
compiled with default configure settings. Overall process user time is measured
via the \code{time} utility. Each benchmark/configuration is run 10 times (after an
initial warmup run to ensure there is an up-to-date bytecode cache file and
compilation time will not be included), and all individual data points are
recorded. All data points are in seconds.

JIT runs use a JIT list that includes only the benchmark code itself, to
minimize overhead of compiling unrelated standard library code not used in the
benchmark.

Benchmarks \bmname{richards} and \bmname{fannkuch} ran on the following commit:

\begin{center}
\footnotesize\url{https://github.com/facebookincubator/cinder/commit/d0d071a9acf3e65700e7c6f8982c5087c700d116}
\end{center}

\noindent\bmname{deltablue} ran on a later commit:

\begin{center}
\footnotesize\url{https://github.com/facebookincubator/cinder/commit/c9d14c4474facfeca15b618015b58b99a3e86d25}
\end{center}

Fannkuch in particular is very numeric-heavy, so using unboxed primitive
integers in the typed version can be a big win. For this to work maximally
efficiently, we also need to avoid operations that don't yet support primitives
and thus would require expensive boxing. In typed-and-optimized Fannkuch (\colname{T-Max}), these changes move
a lot of the work out of heavily-optimized C builtin methods (for list
slice/insert/pop) and into simple Python code doing the equivalent operations
manually with simple iteration and indexing. In the short run this is a massive
pessimization for non-static Python (or even SP without the JIT, since
currently we always use boxed integers in the interpreter loop.) In the long
run it suggests that Static Python + JIT could make it feasible to implement
many more of these core datastructure operations in pure Python rather than in
C.

Because of this subtlety, we include benchmark results for both typed and
untyped versions of each benchmark. In the short term, perhaps the most
relevant comparison is the performance of Static Python on the typed benchmark
vs the performance of nonstatic Python on the untyped benchmark; this gives a
fair picture of expected perf gains converting code to Static Python with some
willingness to optimize.

We also observe significant untapped opportunity to improve Static Python's
performance; for example, $20\%$ of Typed Fannkuch SP+JIT time is spent on
bounds-checking array accesses; most of this runtime bounds-checking cost could
be eliminated if the Static Python compiler tracks known array sizes and
integer values more thoroughly. And it would also be possible to support list
slicing, list pop, and list insert with primitive arguments, thus avoiding the
need to rewrite these operations in Python for maximum efficiency.

\colname{T-Min} Fannkuch without the JIT is slow because of a known issue where we
don't cache function lookups for \code{INVOKE\_FUNCTION} in the interpreter loop,
making the calls to \code{list.insert} and \code{list.pop} slower; this is fixable.

\begin{table}[tp]
  \caption{Richards microbenchmark data}
  \label{t:mb:richards}
  \footnotesize \centering

\begin{tabular}{rrrrrr}
  \colname{T-Max} & \colname{T-Max} & \colname{T-Max} & \colname{T-Max} & \colname{T-Max} & \colname{T-Max} \\
  (\colname{SP} \colname{JIT} \colname{SF}) & (\colname{SP} \colname{JIT}) & (\colname{SP}) & (\colname{JIT} \colname{SF}) & (\colname{JIT}) & () \\\hline
  $0.57$ & $1.29$ & $7.86$ & $8.36$ & $10.24$ & $16$ \\
  $0.59$ & $1.32$ & $7.92$ & $8.84$ & $9.97$ & $15.73$ \\
  $0.58$ & $1.33$ & $7.89$ & $8.9$ & $10.57$ & $16.2$ \\
  $0.58$ & $1.33$ & $7.92$ & $8.53$ & $10.71$ & $16.71$ \\
  $0.56$ & $1.33$ & $7.83$ & $9.15$ & $9.89$ & $16.62$ \\
  $0.56$ & $1.33$ & $7.97$ & $8.25$ & $10.44$ & $15.65$ \\
  $0.55$ & $1.32$ & $8.18$ & $8.78$ & $10.07$ & $16.22$ \\
  $0.55$ & $1.31$ & $7.85$ & $8.67$ & $10.07$ & $15.73$ \\
  $0.56$ & $1.31$ & $7.96$ & $8.52$ & $10.14$ & $15.82$ \\
  $0.57$ & $1.32$ & $8$ & $8.35$ & $10.57$ & $16.66$ \\
\end{tabular}

\begin{tabular}{rrrrrr}
  \colname{T-Min} & \colname{T-Min} & \colname{T-Min} & \colname{T-Min} & \colname{T-Min} & \colname{T-Min} \\
  (\colname{SP} \colname{JIT} \colname{SF}) & (\colname{SP} \colname{JIT}) & (\colname{SP}) & (\colname{JIT} \colname{SF}) & (\colname{JIT}) & () \\\hline
  $1.41$ & $2.68$ & $7.67$ & $5.58$ & $7.97$ & $13.32$ \\
  $1.38$ & $2.6$ & $7.9$ & $5.92$ & $7.25$ & $13.59$ \\
  $1.39$ & $2.64$ & $7.74$ & $5.54$ & $7.64$ & $13.83$ \\
  $1.38$ & $2.65$ & $7.8$ & $6.06$ & $7.24$ & $13.95$ \\
  $1.37$ & $2.62$ & $7.68$ & $5.52$ & $7.2$ & $13.5$ \\
  $1.39$ & $2.58$ & $7.8$ & $5.63$ & $7.27$ & $13.52$ \\
  $1.36$ & $2.65$ & $7.79$ & $5.74$ & $7.17$ & $13.34$ \\
  $1.43$ & $2.66$ & $7.82$ & $5.56$ & $7.54$ & $13.68$ \\
  $1.53$ & $2.63$ & $7.9$ & $5.59$ & $6.89$ & $13.79$ \\
  $1.37$ & $2.71$ & $7.95$ & $5.74$ & $6.91$ & $13.5$ \\
\end{tabular}

\begin{tabular}{rrr}
  \colname{Orig} & \colname{Orig} & \colname{Orig} \\
  (\colname{JIT} \colname{SF}) & (\colname{JIT}) & () \\\hline
  $2.6$ & $3.88$ & $10.17$ \\
  $2.6$ & $4.01$ & $10.18$ \\
  $2.61$ & $4.17$ & $10.73$ \\
  $2.9$ & $3.92$ & $9.96$ \\
  $2.63$ & $3.79$ & $10.04$ \\
  $2.64$ & $3.95$ & $9.99$ \\
  $2.62$ & $3.9$ & $10.16$ \\
  $2.6$ & $3.84$ & $9.69$ \\
  $2.6$ & $4.04$ & $9.82$ \\
  $2.62$ & $3.85$ & $9.85$ \\
\end{tabular}

\end{table}

\begin{table}[tp]
  \caption{Fannkuch microbenchmark data}
  \label{t:mb:fannkuch}
  \footnotesize\centering

  \begin{tabular}{rrrrrr}
  \colname{T-Max} & \colname{T-Max} & \colname{T-Max} & \colname{T-Max} & \colname{T-Max} & \colname{T-Max} \\
  (\colname{SP} \colname{JIT} \colname{SF}) & (\colname{SP} \colname{JIT}) & (\colname{SP}) & (\colname{JIT} \colname{SF}) & (\colname{JIT}) & () \\\hline
  $1.27$ & $1.39$ & $34.69$ & $40.74$ & $41$ & $48.23$ \\
  $1.32$ & $1.25$ & $34.35$ & $40.23$ & $40.94$ & $49.03$ \\
  $1.28$ & $1.26$ & $34.67$ & $40.74$ & $40.65$ & $49.42$ \\
  $1.23$ & $1.31$ & $35.71$ & $40.94$ & $39.7$ & $49.35$ \\
  $1.28$ & $1.38$ & $34.85$ & $39.59$ & $40.46$ & $49.18$ \\
  $1.26$ & $1.34$ & $33.96$ & $40.45$ & $39.99$ & $47.79$ \\
  $1.26$ & $1.23$ & $34.65$ & $40.78$ & $40.97$ & $49.7$ \\
  $1.27$ & $1.35$ & $33.83$ & $40.96$ & $41.78$ & $47.92$ \\
  $1.23$ & $1.4$ & $36.23$ & $40.49$ & $40.29$ & $48.68$ \\
  $1.23$ & $1.27$ & $36.05$ & $40.72$ & $40.11$ & $48.29$ \\
\end{tabular}

\begin{tabular}{rrrrrr}
  \colname{T-Min} & \colname{T-Min} & \colname{T-Min} & \colname{T-Min} & \colname{T-Min} & \colname{T-Min} \\
  (\colname{SP} \colname{JIT} \colname{SF}) & (\colname{SP} \colname{JIT}) & (\colname{SP}) & (\colname{JIT} \colname{SF}) & (\colname{JIT}) & () \\\hline
  $2.77$ & $2.83$ & $6.18$ & $2.89$ & $2.86$ & $4.07$ \\
  $2.85$ & $2.8$ & $6.12$ & $2.94$ & $2.9$ & $4.06$ \\
  $2.79$ & $2.77$ & $6.38$ & $2.87$ & $2.86$ & $4.1$ \\
  $2.87$ & $2.81$ & $6.23$ & $2.87$ & $2.85$ & $4.29$ \\
  $2.85$ & $2.84$ & $6.17$ & $2.92$ & $2.81$ & $4.14$ \\
  $2.87$ & $2.8$ & $6.21$ & $2.94$ & $2.8$ & $4.06$ \\
  $2.92$ & $2.76$ & $6.04$ & $3.11$ & $3.05$ & $4.08$ \\
  $2.78$ & $2.76$ & $6.13$ & $3.1$ & $2.86$ & $4.19$ \\
  $2.84$ & $2.76$ & $6.13$ & $2.87$ & $2.96$ & $4.31$ \\
  $2.92$ & $2.79$ & $6.17$ & $2.91$ & $2.96$ & $4.1$ \\
\end{tabular}

\begin{tabular}{rrrrrr}
  \colname{T-Min-2} & \colname{T-Min-2} & \colname{T-Min-2} & \colname{T-Min-2} & \colname{T-Min-2} & \colname{T-Min-2} \\
  (\colname{SP} \colname{JIT} \colname{SF}) & (\colname{SP} \colname{JIT}) & (\colname{SP}) & (\colname{JIT} \colname{SF}) & (\colname{JIT}) & () \\\hline
  $2.48$ & $2.52$ & $7.28$ & $3.16$ & $3.08$ & $4.59$ \\
  $2.52$ & $2.48$ & $7.28$ & $3.26$ & $3.08$ & $4.51$ \\
  $2.47$ & $2.53$ & $7.24$ & $3.11$ & $3.18$ & $4.59$ \\
  $2.48$ & $2.57$ & $7.35$ & $3.18$ & $3.14$ & $4.73$ \\
  $2.5$ & $2.56$ & $7.34$ & $3.14$ & $3.12$ & $4.63$ \\
  $2.43$ & $2.47$ & $7.17$ & $3.23$ & $3.09$ & $4.53$ \\
  $2.47$ & $2.69$ & $7.18$ & $3.41$ & $3.21$ & $4.48$ \\
  $2.47$ & $2.45$ & $7.14$ & $3.43$ & $3.14$ & $4.53$ \\
  $2.43$ & $2.55$ & $7.23$ & $3.08$ & $3.24$ & $4.53$ \\
  $2.52$ & $2.49$ & $7.2$ & $3.16$ & $3.18$ & $4.45$ \\
\end{tabular}

\begin{tabular}{rrr}
  \colname{Orig} & \colname{Orig} & \colname{Orig} \\
  (\colname{JIT} \colname{SF}) & (\colname{JIT}) & () \\\hline
  $2.81$ & $2.94$ & $4.02$ \\
  $2.66$ & $2.85$ & $4.22$ \\
  $2.69$ & $2.78$ & $4.01$ \\
  $2.75$ & $2.8$ & $4.01$ \\
  $2.75$ & $2.92$ & $4.07$ \\
  $2.78$ & $2.83$ & $4.36$ \\
  $2.84$ & $2.81$ & $4.19$ \\
  $2.7$ & $2.86$ & $4.25$ \\
  $2.76$ & $2.78$ & $4.08$ \\
  $2.89$ & $2.76$ & $4.16$ \\
\end{tabular}

\end{table}

\begin{table}[tp]
  \caption{DeltaBlue microbenchmark data}
  \label{t:mb:deltablue}
  \footnotesize\centering

  \begin{tabular}{rrrrrr}
    \colname{T-Max} & \colname{T-Max} & \colname{T-Max} & \colname{T-Max} & \colname{T-Max} & \colname{T-Max} \\
    (\colname{SP} \colname{JIT} \colname{SF}) & (\colname{SP} \colname{JIT}) & (\colname{SP}) & (\colname{JIT} \colname{SF}) & (\colname{JIT}) & () \\\hline
    $0.3$ & $0.4$ & $1.32$ & $0.51$ & $0.64$ & $1.02$ \\
    $0.32$ & $0.39$ & $1.23$ & $0.54$ & $0.64$ & $1$ \\
    $0.31$ & $0.4$ & $1.24$ & $0.54$ & $0.64$ & $0.99$ \\
    $0.3$ & $0.42$ & $1.31$ & $0.56$ & $0.65$ & $1.02$ \\
    $0.28$ & $0.4$ & $1.24$ & $0.52$ & $0.65$ & $0.96$ \\
    $0.28$ & $0.39$ & $1.24$ & $0.55$ & $0.65$ & $0.98$ \\
    $0.29$ & $0.4$ & $1.27$ & $0.55$ & $0.68$ & $0.97$ \\
    $0.3$ & $0.42$ & $1.29$ & $0.53$ & $0.67$ & $1$ \\
    $0.29$ & $0.39$ & $1.28$ & $0.53$ & $0.69$ & $0.99$ \\
    $0.28$ & $0.42$ & $1.32$ & $0.51$ & $0.64$ & $0.99$ \\
  \end{tabular}

  \begin{tabular}{rrrrrr}
    \colname{T-Min} & \colname{T-Min} & \colname{T-Min} & \colname{T-Min} & \colname{T-Min} & \colname{T-Min} \\
    (\colname{SP} \colname{JIT} \colname{SF}) & (\colname{SP} \colname{JIT}) & (\colname{SP}) & (\colname{JIT} \colname{SF}) & (\colname{JIT}) & () \\\hline
    $0.6$ & $0.74$ & $1.32$ & $0.69$ & $0.82$ & $1.27$ \\
    $0.59$ & $0.74$ & $1.32$ & $0.67$ & $0.84$ & $1.27$ \\
    $0.57$ & $0.71$ & $1.31$ & $0.7$ & $0.89$ & $1.23$ \\
    $0.58$ & $0.74$ & $1.23$ & $0.73$ & $0.82$ & $1.26$ \\
    $0.6$ & $0.71$ & $1.32$ & $0.71$ & $0.84$ & $1.26$ \\
    $0.59$ & $0.73$ & $1.4$ & $0.71$ & $0.84$ & $1.23$ \\
    $0.56$ & $0.7$ & $1.37$ & $0.71$ & $0.85$ & $1.35$ \\
    $0.58$ & $0.69$ & $1.3$ & $0.71$ & $0.82$ & $1.31$ \\
    $0.58$ & $0.69$ & $1.27$ & $0.69$ & $0.81$ & $1.34$ \\
    $0.6$ & $0.73$ & $1.39$ & $0.68$ & $0.8$ & $1.33$ \\
  \end{tabular}

  \begin{tabular}{rrrrrr}
    \colname{T-Min-2} & \colname{T-Min-2} & \colname{T-Min-2} & \colname{T-Min-2} & \colname{T-Min-2} & \colname{T-Min-2} \\
    (\colname{SP} \colname{JIT} \colname{SF}) & (\colname{SP} \colname{JIT}) & (\colname{SP}) & (\colname{JIT} \colname{SF}) & (\colname{JIT}) & () \\\hline
    $0.54$ & $0.66$ & $1.38$ & $0.61$ & $0.76$ & $1.17$ \\
    $0.51$ & $0.62$ & $1.44$ & $0.61$ & $0.75$ & $1.25$ \\
    $0.55$ & $0.69$ & $1.41$ & $0.6$ & $0.82$ & $1.17$ \\
    $0.54$ & $0.69$ & $1.41$ & $0.63$ & $0.75$ & $1.16$ \\
    $0.54$ & $0.7$ & $1.41$ & $0.65$ & $0.75$ & $1.17$ \\
    $0.55$ & $0.67$ & $1.4$ & $0.66$ & $0.75$ & $1.15$ \\
    $0.52$ & $0.67$ & $1.43$ & $0.61$ & $0.75$ & $1.15$ \\
    $0.52$ & $0.66$ & $1.4$ & $0.62$ & $0.75$ & $1.18$ \\
    $0.52$ & $0.64$ & $1.51$ & $0.65$ & $0.76$ & $1.18$ \\
    $0.5$ & $0.65$ & $1.41$ & $0.62$ & $0.79$ & $1.15$ \\
  \end{tabular}

  \begin{tabular}{rrr}
    \colname{Orig} & \colname{Orig} & \colname{Orig} \\
    (\colname{JIT} \colname{SF}) & (\colname{JIT}) & () \\\hline
    $0.64$ & $0.78$ & $1.22$ \\
    $0.64$ & $0.81$ & $1.25$ \\
    $0.63$ & $0.81$ & $1.23$ \\
    $0.62$ & $0.79$ & $1.2$ \\
    $0.67$ & $0.81$ & $1.23$ \\
    $0.7$ & $0.8$ & $1.23$ \\
    $0.68$ & $0.78$ & $1.34$ \\
    $0.66$ & $0.79$ & $1.22$ \\
    $0.69$ & $0.85$ & $1.23$ \\
    $0.68$ & $0.79$ & $1.24$ \\
  \end{tabular}
\end{table}

\begin{table}[tp]
  \caption{Nbody microbenchmark data}
  \label{t:mb:nbody}
  \footnotesize\centering

  \begin{tabular}{rrrrrr}
    \colname{T-Max} & \colname{T-Max} & \colname{T-Max} & \colname{T-Max} & \colname{T-Max} & \colname{T-Max} \\
    (\colname{SP} \colname{JIT} \colname{SF}) & (\colname{SP} \colname{JIT}) & (\colname{SP}) & (\colname{JIT} \colname{SF}) & (\colname{JIT}) & () \\\hline
    $0.17$ & $0.18$ & $3.61$ & $0.92$ & $0.88$ & $2.12$ \\
    $0.17$ & $0.18$ & $3.57$ & $0.91$ & $0.88$ & $2.09$ \\
    $0.17$ & $0.18$ & $3.68$ & $0.96$ & $0.88$ & $2.24$ \\
    $0.17$ & $0.17$ & $3.82$ & $0.93$ & $0.9$ & $2.15$ \\
    $0.17$ & $0.17$ & $3.72$ & $0.89$ & $0.9$ & $2.17$ \\
    $0.18$ & $0.18$ & $3.63$ & $0.89$ & $0.9$ & $2.14$ \\
    $0.18$ & $0.16$ & $3.5$ & $0.89$ & $0.89$ & $2.23$ \\
    $0.18$ & $0.17$ & $3.45$ & $0.89$ & $0.93$ & $2.32$ \\
    $0.2$ & $0.17$ & $3.61$ & $0.92$ & $0.88$ & $2.16$ \\
    $0.2$ & $0.18$ & $3.63$ & $0.9$ & $0.92$ & $2.06$ \\
  \end{tabular}

  \begin{tabular}{rrrrrr}
    \colname{T-Min} & \colname{T-Min} & \colname{T-Min} & \colname{T-Min} & \colname{T-Min} & \colname{T-Min} \\
    (\colname{SP} \colname{JIT} \colname{SF}) & (\colname{SP} \colname{JIT}) & (\colname{SP}) & (\colname{JIT} \colname{SF}) & (\colname{JIT}) & () \\\hline
    $0.8$ & $0.81$ & $1.3$ & $0.83$ & $0.74$ & $1.2$ \\
    $0.8$ & $0.8$ & $1.28$ & $0.78$ & $0.74$ & $1.22$ \\
    $0.81$ & $0.8$ & $1.31$ & $0.75$ & $0.77$ & $1.22$ \\
    $0.87$ & $0.86$ & $1.29$ & $0.74$ & $0.76$ & $1.18$ \\
    $0.85$ & $0.8$ & $1.26$ & $0.76$ & $0.77$ & $1.19$ \\
    $0.81$ & $0.81$ & $1.3$ & $0.77$ & $0.75$ & $1.22$ \\
    $0.86$ & $0.79$ & $1.25$ & $0.76$ & $0.75$ & $1.17$ \\
    $0.87$ & $0.83$ & $1.24$ & $0.74$ & $0.77$ & $1.16$ \\
    $0.81$ & $0.81$ & $1.3$ & $0.74$ & $0.74$ & $1.17$ \\
    $0.82$ & $0.83$ & $1.23$ & $0.76$ & $0.76$ & $1.16$ \\
  \end{tabular}

  \begin{tabular}{rrr}
    \colname{Orig} & \colname{Orig} & \colname{Orig} \\
    (\colname{JIT} \colname{SF}) & (\colname{JIT}) & () \\\hline
    $0.78$ & $0.75$ & $1.15$ \\
    $0.74$ & $0.73$ & $1.19$ \\
    $0.75$ & $0.72$ & $1.19$ \\
    $0.73$ & $0.73$ & $1.18$ \\
    $0.83$ & $0.76$ & $1.16$ \\
    $0.82$ & $0.72$ & $1.24$ \\
    $0.75$ & $0.72$ & $1.15$ \\
    $0.72$ & $0.76$ & $1.2$ \\
    $0.72$ & $0.72$ & $1.15$ \\
    $0.74$ & $0.73$ & $1.2$ \\
  \end{tabular}
\end{table}


\begin{table}[tp]
  \caption{Exact commands to invoke each benchmark (1/2)}
  \label{t:mb:commands1}
  \tiny\centering

  \begin{tabular}{ll}
    \colname{Benchmark} & \colname{Command} \\\hline
    Richards \colname{T-Max} & \texttt{time ./python -X jit -X jit-list-file=Tools/benchmarks/jitlist\_richards\_static.txt -X jit-enable-jit-list-wildcards} \\
    (\colname{SP} \colname{JIT} \colname{SF}) & \texttt{ -X jit-shadow-frame -X install-strict-loader Tools/benchmarks/richards\_static.py 100} \\
    Richards \colname{T-Max} & \texttt{time ./python -X jit -X jit-list-file=Tools/benchmarks/jitlist\_richards\_static.txt -X jit-enable-jit-list-wildcards} \\
    (\colname{SP} \colname{JIT}) & \texttt{ -X install-strict-loader Tools/benchmarks/richards\_static.py 100} \\
    Richards \colname{T-Max} & \texttt{time ./python -X install-strict-loader Tools/benchmarks/richards\_static.py 100} \\
    (\colname{SP}) &  \\
    Richards \colname{T-Max} & \texttt{time ./python -X jit -X jit-list-file=Tools/benchmarks/jitlist\_richards\_static.txt -X jit-enable-jit-list-wildcards} \\
    (\colname{JIT} \colname{SF}) & \texttt{ -X jit-shadow-frame Tools/benchmarks/richards\_static.py 100} \\
    Richards \colname{T-Max} & \texttt{time ./python -X jit -X jit-list-file=Tools/benchmarks/jitlist\_richards\_static.txt -X jit-enable-jit-list-wildcards} \\
    (\colname{JIT}) & \texttt{ Tools/benchmarks/richards\_static.py 100} \\
    Richards \colname{T-Max} & \texttt{time ./python Tools/benchmarks/richards\_static.py 100} \\
    () &  \\
    Richards \colname{T-Min} & \texttt{time ./python -X jit -X jit-list-file=Tools/benchmarks/jitlist\_richards\_static\_basic.txt -X jit-enable-jit-list-wil} \\
    (\colname{SP} \colname{JIT} \colname{SF}) & \texttt{dcards -X jit-shadow-frame -X install-strict-loader Tools/benchmarks/richards\_static\_basic.py 100} \\
    Richards \colname{T-Min} & \texttt{time ./python -X jit -X jit-list-file=Tools/benchmarks/jitlist\_richards\_static\_basic.txt -X jit-enable-jit-list-wil} \\
    (\colname{SP} \colname{JIT}) & \texttt{dcards -X install-strict-loader Tools/benchmarks/richards\_static\_basic.py 100} \\
    Richards \colname{T-Min} & \texttt{time ./python -X install-strict-loader Tools/benchmarks/richards\_static\_basic.py 100} \\
    (\colname{SP}) &  \\
    Richards \colname{T-Min} & \texttt{time ./python -X jit -X jit-list-file=Tools/benchmarks/jitlist\_richards\_static\_basic.txt -X jit-enable-jit-list-wil} \\
    (\colname{JIT} \colname{SF}) & \texttt{dcards -X jit-shadow-frame Tools/benchmarks/richards\_static\_basic.py 100} \\
    Richards \colname{T-Min} & \texttt{time ./python -X jit -X jit-list-file=Tools/benchmarks/jitlist\_richards\_static\_basic.txt -X jit-enable-jit-list-wil} \\
    (\colname{JIT}) & \texttt{dcards Tools/benchmarks/richards\_static\_basic.py 100} \\
    Richards \colname{T-Min} & \texttt{time ./python Tools/benchmarks/richards\_static\_basic.py 100} \\
    () &  \\
    Richards \colname{Orig} & \texttt{time ./python -X jit -X jit-list-file=Tools/benchmarks/jitlist\_main.txt -X jit-enable-jit-list-wildcards -X jit-sha} \\
    (\colname{JIT} \colname{SF}) & \texttt{dow-frame Tools/benchmarks/richards.py 100} \\
    Richards \colname{Orig} & \texttt{time ./python -X jit -X jit-list-file=Tools/benchmarks/jitlist\_main.txt -X jit-enable-jit-list-wildcards Tools/benc} \\
    (\colname{JIT}) & \texttt{hmarks/richards.py 100} \\
    Richards \colname{Orig} & \texttt{time ./python Tools/benchmarks/richards.py 100} \\
    () &  \\
    Fannkuch \colname{T-Max} & \texttt{time ./python -X jit -X jit-list-file=Tools/benchmarks/jitlist\_fannkuch\_static.txt -X jit-enable-jit-list-wildcards} \\
    (\colname{SP} \colname{JIT} \colname{SF}) & \texttt{ -X jit-shadow-frame -X install-strict-loader Tools/benchmarks/fannkuch\_static.py 5} \\
    Fannkuch \colname{T-Max} & \texttt{time ./python -X jit -X jit-list-file=Tools/benchmarks/jitlist\_fannkuch\_static.txt -X jit-enable-jit-list-wildcards} \\
    (\colname{SP} \colname{JIT}) & \texttt{ -X install-strict-loader Tools/benchmarks/fannkuch\_static.py 5} \\
    Fannkuch \colname{T-Max} & \texttt{time ./python -X install-strict-loader Tools/benchmarks/fannkuch\_static.py 5} \\
    (\colname{SP}) &  \\
    Fannkuch \colname{T-Max} & \texttt{time ./python -X jit -X jit-list-file=Tools/benchmarks/jitlist\_fannkuch\_static.txt -X jit-enable-jit-list-wildcards} \\
    (\colname{JIT} \colname{SF}) & \texttt{ -X jit-shadow-frame Tools/benchmarks/fannkuch\_static.py 5} \\
    Fannkuch \colname{T-Max} & \texttt{time ./python -X jit -X jit-list-file=Tools/benchmarks/jitlist\_fannkuch\_static.txt -X jit-enable-jit-list-wildcards} \\
    (\colname{JIT}) & \texttt{ Tools/benchmarks/fannkuch\_static.py 5} \\
    Fannkuch \colname{T-Max} & \texttt{time ./python Tools/benchmarks/fannkuch\_static.py 5} \\
    () &  \\
    Fannkuch \colname{T-Min} & \texttt{time ./python -X jit -X jit-list-file=Tools/benchmarks/jitlist\_fannkuch\_static\_basic.txt -X jit-enable-jit-list-wil} \\
    (\colname{SP} \colname{JIT} \colname{SF}) & \texttt{dcards -X jit-shadow-frame -X install-strict-loader Tools/benchmarks/fannkuch\_static\_basic.py 5} \\
    Fannkuch \colname{T-Min} & \texttt{time ./python -X jit -X jit-list-file=Tools/benchmarks/jitlist\_fannkuch\_static\_basic.txt -X jit-enable-jit-list-wil} \\
    (\colname{SP} \colname{JIT}) & \texttt{dcards -X install-strict-loader Tools/benchmarks/fannkuch\_static\_basic.py 5} \\
    Fannkuch \colname{T-Min} & \texttt{time ./python -X install-strict-loader Tools/benchmarks/fannkuch\_static\_basic.py 5} \\
    (\colname{SP}) &  \\
    Fannkuch \colname{T-Min} & \texttt{time ./python -X jit -X jit-list-file=Tools/benchmarks/jitlist\_fannkuch\_static\_basic.txt -X jit-enable-jit-list-wil} \\
    (\colname{JIT} \colname{SF}) & \texttt{dcards -X jit-shadow-frame Tools/benchmarks/fannkuch\_static\_basic.py 5} \\
    Fannkuch \colname{T-Min} & \texttt{time ./python -X jit -X jit-list-file=Tools/benchmarks/jitlist\_fannkuch\_static\_basic.txt -X jit-enable-jit-list-wil} \\
    (\colname{JIT}) & \texttt{dcards Tools/benchmarks/fannkuch\_static\_basic.py 5} \\
    Fannkuch \colname{T-Min} & \texttt{time ./python Tools/benchmarks/fannkuch\_static\_basic.py 5} \\
    () &  \\
    Fannkuch \colname{T-Min} & \texttt{time ./python -X jit -X jit-list-file=Tools/benchmarks/jitlist\_fannkuch\_static\_basic2.txt -X jit-enable-jit-list-wi} \\
    (\colname{SP} \colname{JIT} \colname{SF}) & \texttt{ldcards -X jit-shadow-frame -X install-strict-loader Tools/benchmarks/fannkuch\_static\_basic2.py 5} \\
    Fannkuch \colname{T-Min} & \texttt{time ./python -X jit -X jit-list-file=Tools/benchmarks/jitlist\_fannkuch\_static\_basic2.txt -X jit-enable-jit-list-wi} \\
    (\colname{SP} \colname{JIT}) & \texttt{ldcards -X install-strict-loader Tools/benchmarks/fannkuch\_static\_basic2.py 5} \\
    Fannkuch \colname{T-Min} & \texttt{time ./python -X install-strict-loader Tools/benchmarks/fannkuch\_static\_basic2.py 5} \\
    (\colname{SP}) &  \\
    Fannkuch \colname{T-Min} & \texttt{time ./python -X jit -X jit-list-file=Tools/benchmarks/jitlist\_fannkuch\_static\_basic2.txt -X jit-enable-jit-list-wi} \\
    (\colname{JIT} \colname{SF}) & \texttt{ldcards -X jit-shadow-frame Tools/benchmarks/fannkuch\_static\_basic2.py 5} \\
    Fannkuch \colname{T-Min} & \texttt{time ./python -X jit -X jit-list-file=Tools/benchmarks/jitlist\_fannkuch\_static\_basic2.txt -X jit-enable-jit-list-wi} \\
    (\colname{JIT}) & \texttt{ldcards Tools/benchmarks/fannkuch\_static\_basic2.py 5} \\
    Fannkuch \colname{T-Min} & \texttt{time ./python Tools/benchmarks/fannkuch\_static\_basic2.py 5} \\
    () &  \\
    Fannkuch \colname{Orig} & \texttt{time ./python -X jit -X jit-list-file=Tools/benchmarks/jitlist\_main.txt -X jit-enable-jit-list-wildcards -X jit-sha} \\
    (\colname{JIT} \colname{SF}) & \texttt{dow-frame Tools/benchmarks/fannkuch.py 5} \\
    Fannkuch \colname{Orig} & \texttt{time ./python -X jit -X jit-list-file=Tools/benchmarks/jitlist\_main.txt -X jit-enable-jit-list-wildcards Tools/benc} \\
    (\colname{JIT}) & \texttt{hmarks/fannkuch.py 5} \\
    Fannkuch \colname{Orig} & \texttt{time ./python Tools/benchmarks/fannkuch.py 5} \\
    () &  \\
  \end{tabular}
\end{table}

\begin{table}[tp]
  \caption{Exact commands to invoke each benchmark (2/2)}
  \label{t:mb:commands2}
  \tiny\centering

  \begin{tabular}{ll}
    \colname{Benchmark} & \colname{Command} \\\hline
    DeltaBlue \colname{T-Max} & \texttt{time ./python -X jit -X jit-list-file=Tools/benchmarks/jitlist\_deltablue\_static.txt -X jit-enable-jit-list-wildcard} \\
    (\colname{SP} \colname{JIT} \colname{SF}) & \texttt{s -X jit-shadow-frame -X install-strict-loader Tools/benchmarks/deltablue\_static.py 100} \\
    DeltaBlue \colname{T-Max} & \texttt{time ./python -X jit -X jit-list-file=Tools/benchmarks/jitlist\_deltablue\_static.txt -X jit-enable-jit-list-wildcard} \\
    (\colname{SP} \colname{JIT}) & \texttt{s -X install-strict-loader Tools/benchmarks/deltablue\_static.py 100} \\
    DeltaBlue \colname{T-Max} & \texttt{time ./python -X install-strict-loader Tools/benchmarks/deltablue\_static.py 100} \\
    (\colname{SP}) &  \\
    DeltaBlue \colname{T-Max} & \texttt{time ./python -X jit -X jit-list-file=Tools/benchmarks/jitlist\_deltablue\_static.txt -X jit-enable-jit-list-wildcard} \\
    (\colname{JIT} \colname{SF}) & \texttt{s -X jit-shadow-frame Tools/benchmarks/deltablue\_static.py 100} \\
    DeltaBlue \colname{T-Max} & \texttt{time ./python -X jit -X jit-list-file=Tools/benchmarks/jitlist\_deltablue\_static.txt -X jit-enable-jit-list-wildcard} \\
    (\colname{JIT}) & \texttt{s Tools/benchmarks/deltablue\_static.py 100} \\
    DeltaBlue \colname{T-Max} & \texttt{time ./python Tools/benchmarks/deltablue\_static.py 100} \\
    () &  \\
    DeltaBlue \colname{T-Min} & \texttt{time ./python -X jit -X jit-list-file=Tools/benchmarks/jitlist\_deltablue\_static\_basic.txt -X jit-enable-jit-list-wi} \\
    (\colname{SP} \colname{JIT} \colname{SF}) & \texttt{ldcards -X jit-shadow-frame -X install-strict-loader Tools/benchmarks/deltablue\_static\_basic.py 100} \\
    DeltaBlue \colname{T-Min} & \texttt{time ./python -X jit -X jit-list-file=Tools/benchmarks/jitlist\_deltablue\_static\_basic.txt -X jit-enable-jit-list-wi} \\
    (\colname{SP} \colname{JIT}) & \texttt{ldcards -X install-strict-loader Tools/benchmarks/deltablue\_static\_basic.py 100} \\
    DeltaBlue \colname{T-Min} & \texttt{time ./python -X install-strict-loader Tools/benchmarks/deltablue\_static\_basic.py 100} \\
    (\colname{SP}) &  \\
    DeltaBlue \colname{T-Min} & \texttt{time ./python -X jit -X jit-list-file=Tools/benchmarks/jitlist\_deltablue\_static\_basic.txt -X jit-enable-jit-list-wi} \\
    (\colname{JIT} \colname{SF}) & \texttt{ldcards -X jit-shadow-frame Tools/benchmarks/deltablue\_static\_basic.py 100} \\
    DeltaBlue \colname{T-Min} & \texttt{time ./python -X jit -X jit-list-file=Tools/benchmarks/jitlist\_deltablue\_static\_basic.txt -X jit-enable-jit-list-wi} \\
    (\colname{JIT}) & \texttt{ldcards Tools/benchmarks/deltablue\_static\_basic.py 100} \\
    DeltaBlue \colname{T-Min} & \texttt{time ./python Tools/benchmarks/deltablue\_static\_basic.py 100} \\
    () &  \\
    DeltaBlue \colname{T-Min} & \texttt{time ./python -X jit -X jit-list-file=Tools/benchmarks/jitlist\_deltablue\_static\_basic2.txt -X jit-enable-jit-list-w} \\
    (\colname{SP} \colname{JIT} \colname{SF}) & \texttt{ildcards -X jit-shadow-frame -X install-strict-loader Tools/benchmarks/deltablue\_static\_basic2.py 100} \\
    DeltaBlue \colname{T-Min} & \texttt{time ./python -X jit -X jit-list-file=Tools/benchmarks/jitlist\_deltablue\_static\_basic2.txt -X jit-enable-jit-list-w} \\
    (\colname{SP} \colname{JIT}) & \texttt{ildcards -X install-strict-loader Tools/benchmarks/deltablue\_static\_basic2.py 100} \\
    DeltaBlue \colname{T-Min} & \texttt{time ./python -X install-strict-loader Tools/benchmarks/deltablue\_static\_basic2.py 100} \\
    (\colname{SP}) &  \\
    DeltaBlue \colname{T-Min} & \texttt{time ./python -X jit -X jit-list-file=Tools/benchmarks/jitlist\_deltablue\_static\_basic2.txt -X jit-enable-jit-list-w} \\
    (\colname{JIT} \colname{SF}) & \texttt{ildcards -X jit-shadow-frame Tools/benchmarks/deltablue\_static\_basic2.py 100} \\
    DeltaBlue \colname{T-Min} & \texttt{time ./python -X jit -X jit-list-file=Tools/benchmarks/jitlist\_deltablue\_static\_basic2.txt -X jit-enable-jit-list-w} \\
    (\colname{JIT}) & \texttt{ildcards Tools/benchmarks/deltablue\_static\_basic2.py 100} \\
    DeltaBlue \colname{T-Min} & \texttt{time ./python Tools/benchmarks/deltablue\_static\_basic2.py 100} \\
    () &  \\
    DeltaBlue \colname{Orig} & \texttt{time ./python -X jit -X jit-list-file=Tools/benchmarks/jitlist\_main.txt -X jit-enable-jit-list-wildcards -X jit-sha} \\
    (\colname{JIT} \colname{SF}) & \texttt{dow-frame Tools/benchmarks/deltablue.py 100} \\
    DeltaBlue \colname{Orig} & \texttt{time ./python -X jit -X jit-list-file=Tools/benchmarks/jitlist\_main.txt -X jit-enable-jit-list-wildcards Tools/benc} \\
    (\colname{JIT}) & \texttt{hmarks/deltablue.py 100} \\
    DeltaBlue \colname{Orig} & \texttt{time ./python Tools/benchmarks/deltablue.py 100} \\
    () &  \\
    NBody \colname{T-Max} & \texttt{time ./python -X jit -X jit-list-file=Tools/benchmarks/jitlist\_nbody\_static.txt -X jit-enable-jit-list-wildcards -X} \\
    (\colname{SP} \colname{JIT} \colname{SF}) & \texttt{ jit-shadow-frame -X install-strict-loader Tools/benchmarks/nbody\_static.py} \\
    NBody \colname{T-Max} & \texttt{time ./python -X jit -X jit-list-file=Tools/benchmarks/jitlist\_nbody\_static.txt -X jit-enable-jit-list-wildcards -X} \\
    (\colname{SP} \colname{JIT}) & \texttt{ install-strict-loader Tools/benchmarks/nbody\_static.py} \\
    NBody \colname{T-Max} & \texttt{time ./python -X install-strict-loader Tools/benchmarks/nbody\_static.py} \\
    (\colname{SP}) &  \\
    NBody \colname{T-Max} & \texttt{time ./python -X jit -X jit-list-file=Tools/benchmarks/jitlist\_nbody\_static.txt -X jit-enable-jit-list-wildcards -X} \\
    (\colname{JIT} \colname{SF}) & \texttt{ jit-shadow-frame Tools/benchmarks/nbody\_static.py} \\
    NBody \colname{T-Max} & \texttt{time ./python -X jit -X jit-list-file=Tools/benchmarks/jitlist\_nbody\_static.txt -X jit-enable-jit-list-wildcards To} \\
    (\colname{JIT}) & \texttt{ols/benchmarks/nbody\_static.py} \\
    NBody \colname{T-Max} & \texttt{time ./python Tools/benchmarks/nbody\_static.py} \\
    () &  \\
    NBody \colname{T-Min} & \texttt{time ./python -X jit -X jit-list-file=Tools/benchmarks/jitlist\_nbody\_static\_basic.txt -X jit-enable-jit-list-wildca} \\
    (\colname{SP} \colname{JIT} \colname{SF}) & \texttt{rds -X jit-shadow-frame -X install-strict-loader Tools/benchmarks/nbody\_static\_basic.py} \\
    NBody \colname{T-Min} & \texttt{time ./python -X jit -X jit-list-file=Tools/benchmarks/jitlist\_nbody\_static\_basic.txt -X jit-enable-jit-list-wildca} \\
    (\colname{SP} \colname{JIT}) & \texttt{rds -X install-strict-loader Tools/benchmarks/nbody\_static\_basic.py} \\
    NBody \colname{T-Min} & \texttt{time ./python -X install-strict-loader Tools/benchmarks/nbody\_static.py} \\
    (\colname{SP}) &  \\
    NBody \colname{T-Min} & \texttt{time ./python -X jit -X jit-list-file=Tools/benchmarks/jitlist\_nbody\_static\_basic.txt -X jit-enable-jit-list-wildca} \\
    (\colname{JIT} \colname{SF}) & \texttt{rds -X jit-shadow-frame Tools/benchmarks/nbody\_static\_basic.py} \\
    NBody \colname{T-Min} & \texttt{time ./python -X jit -X jit-list-file=Tools/benchmarks/jitlist\_nbody\_static.txt -X jit-enable-jit-list-wildcards To} \\
    (\colname{JIT}) & \texttt{ols/benchmarks/nbody\_static.py} \\
    NBody \colname{T-Min} & \texttt{time ./python Tools/benchmarks/nbody\_static\_basic.py} \\
    () &  \\
    NBody \colname{Orig} & \texttt{time ./python -X jit -X jit-list-file=Tools/benchmarks/jitlist\_main.txt -X jit-enable-jit-list-wildcards -X jit-sha} \\
    (\colname{JIT} \colname{SF}) & \texttt{dow-frame Tools/benchmarks/nbody.py} \\
    NBody \colname{Orig} & \texttt{time ./python -X jit -X jit-list-file=Tools/benchmarks/jitlist\_main.txt -X jit-enable-jit-list-wildcards Tools/benc} \\
    (\colname{JIT}) & \texttt{hmarks/nbody.py} \\
    NBody \colname{Orig} & \texttt{time ./python Tools/benchmarks/nbody.py} \\
    () &  \\
  \end{tabular}

\end{table}


\end{document}
